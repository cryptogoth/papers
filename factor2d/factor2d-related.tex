\section{Related Work}
\label{sec:related}

We extend the body of work which applies classical ideas to
quantum logic. Gossett \cite{Gossett1998} uses carry-save techniques to add
numbers in constant-depth and multiply in logarithmic-depth
using a special encoding, but at a quadratic
cost in qubits (circuit width). The underlying idea of encoded adding, sometimes
called a 3-2 adder, derives from Wallace trees \cite{Wallace1964}.

Choi and Van Meter are the first to discuss 2D architectures by designing an
adder that runs in $\Theta(\sqrt{n})$-depth on \textsc{2D NTC} \cite{Choi2010}
using $O(n)$-qubits with dedicated, special-purpose areas of a physical
circuit layout.

Takahashi and Kunihiro have also discovered a linear-depth
and linear-size adder using zero ancillae \cite{Takahashi2005}, and also
an adder with variable tradeoffs between $O(n/d(n))$ ancillae and
$O(d(n))$-depth for $d(n) = \Omega(\log n)$ \cite{Takahashi2009} which has
better width but worse depth than our adder. This approach assumes unbounded
fanout, which has not been mapped to a
nearest-neighbor circuit until the current work.

Once an adder implementation is chosen, it can be extended to perform
modular reduction, modular multiplication, modular
exponentiation, and ultimately
quantum period finding (QPF), the only quantum part of the factoring algorithm.
Since Shor's algorithm is a probabilistic algorithm, requiring several rounds of 
QPF to amplify success probability, it suffices to determine the resources
required for a single round of QPF with a fixed, modest success probability.
The original approach to QPF performs controlled
modular exponentiation followed by an inverse quantum Fourier transform
(QFT) \cite{Nielsen2000}. We will call this \emph{serial QPF}.

This is the approach taken by all other factoring (QPF)
implementations on any architectural model before the current work.
For example, Beauregard \cite{Beauregard2002} uses this QPF approach
to construct
a cubic-depth quantum period-finder using only $2n+3$ qubits on AC,
by combining the ideas of Draper's transform adder \cite{Draper2000},
Vedral et al.'s modular arithmetic blocks \cite{Vedral1996}, and a
semi-classical QFT.
This approach was subsequently adapted to \textsc{1D NTC} by Fowler, Devitt,
and Hollenberg
\cite{Fowler2004} to achieve exact resource counts for an $O(n^3)$-depth
quantum period-finder. Kutin \cite{Kutin2006} later improved this using
an idea from Zalka for approximate multipliers to get a QPF circuit on
\textsc{1D NTC}
in $O(n^2)$-depth. Thus, there is only a constant overhead from
Zalka's own factoring implementation on \textsc{AC}, also in quadratic depth
\cite{Zalka1998}.
Takahashi and Tani extend their earlier $O(n)$-depth adder to a factoring
circuit in $O(n^3)$-depth but with linear width.

All these works assume qubits are expensive (width) and that
execution time (depth) is not the limiting constraint.
We compare our work primarily against Kutin's method, and we make the
alternative assumption that ancillae are cheap and that fast classical control
is available which can access all qubits in parallel. Therefore, we optimize
circuit depth at the expense of width.

Serial QPF is depth-limited by having to the perform an inverse QFT.
On an AC architecture, even when approximating the (inverse) QFT
by truncating two-qubit
$\pi/2^k$ rotations beyond
$k = O(\log n)$, the depth is $O(n \log n)$ for factoring $n$-bit numbers.
There is an alternative, parallel version of phase estimation described in
Section 13 of \cite{Kitaev2002}, which decreases depth in exchange
for increased width and additional classical post-processing.
This eliminates the need to do an inverse QFT.
We refer the reader to \cite{Kitaev2002} and \cite{Pham2011b} for details.
Our factoring scheme employs our 2D quantum arithmetic circuits and this
\emph{parallel QPF}, and we will show that it is asymptotically more
efficient than the other QPF method. We compare the circuit resources
required by our work with the serial QPF implementations above in Table
\ref{tab:results} of Section \ref{sec:results}.

Recent results by Browne, Kashefi, and Perdrix (BKP) connect the power of
measurement-based quantum computing to the quantum circuit model augmented with
unbounded fanout \cite{Browne2009}. Their model, which we adapt and call
\textsc{CCNTC}, uses the classical controller mentioned in \ref{subsec:fanout}.
They describe a constant-depth circuit for
exact factoring, improving on a constant-depth circuit for approximate factoring
by H{\o}yer and {\v S}palek \cite{Hoyer2002}.
A direction for future work is to determine how our approach compares to the
BKP result in terms of circuit size and width.
