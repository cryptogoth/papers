\section{Conclusions and Future Work}
\label{sec:conclude}

In this paper, we have presented a 2D architecture for factoring on a quantum
computer.
%While this is only an intermediate progress report, the preliminary
%results are promising.
Using a combination of algorithmic
improvements (carry-save adders and parallelized phase estimation) 
and architectural improvements (irregular two-dimensional layouts and
constant-depth communication), we conclude
that we can run
the central part of Shor's factoring algorithm (quantum period-finding)
with asymptotically smaller depth than previous implementations.
%, both
%asymptotically and numerically.

%We also provide a constructive upper bound on the overhead
%of nearest-neighbor (NTC) architectures over AC architectures.
%The best-known depth for factoring has constant-depth, but it is an open
%question whether we can satisfy the nearest-neighbor constraint while
%maintaining this constant depth. Furthermore, the cost of decreasing depth
%is often that of increasing width or size, and this time-space tradeoff is
%currently not very well known both for the current work and other
%implementations mentioned in Section \ref{sec:related}.
%Characterizing these tradeoffs is the most obvious extension to
%the current work.

For future work, we would like to determine the exact width, depth, and size of
our proposed factoring circuit, including the constants, as well as
further optimizing our depth to be constant.
%We would also like to determine how our approach compares to the constant-depth
%results in \cite{Browne2009} in terms of circuit size and width.
Along those lines, Rosenbaum recently showed how to convert any $n$-qubit CCAC
circuit to an equivalent CCNTC circuit in constant depth using $n^2$ ancillae
\cite{Rosenbaum2012}.
It is an interesting open question how a
generic conversion of a constant-depth CCAC factoring architecture
\cite{Browne2009,Hoyer2002} to CCNTC compares to our hand-optimized circuit.
The depth of our circuit may also be improved by extending the carry-save adder
from
a $3\rightarrow 2$ circuit to any $2^{n-1}\rightarrow n$ circuit.

The authors wish to thank Aram Harrow, Austin Fowler, and David Rosenbaum for
useful discussions.
P. Pham conducted the factoring part of this work during
an internship at Microsoft Research.
He also acknowledges funding of the architecture and layout portions
of this work from
the Intelligence Advanced Research Projects Activity
(IARPA) via Department of Interior National Business Center contract
number D11PC20167. The U.S. Government is authorized to reproduce and
distribute reprints for Governmental purposes notwithstanding any
copyright annotation thereon. Disclaimer: The views and conclusions
contained herein are those of the authors and should not be
interpreted as necessarily representing the official policies or
endorsements, either expressed or implied, of IARPA, DoI/NBC, or the
U.S. Government.
