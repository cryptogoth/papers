%%%%%%%%%%%%%%%%%%%%%%%%%%%%%%%%%%%%%%%%%%%%%%%%%%%%%%%%%%%%%%%%%%%%%%%%%%%%%%
% This is the LaTeX source for
% "A 2D Nearest-Neighbor Quantum Architecture for Factoring"
% submitted to the Reversible Computing Workshop (RC 2012)
% based on Spring-Verlag's Lecture Notes in Computer Sciences template
% typeinst.tex
% Paul Pham and Krysta Svore
% 14 March 2012
%%%%%%%%%%%%%%%%%%%%%%%%%%%%%%%%%%%%%%%%%%%%%%%%%%%%%%%%%%%%%%%%%%%%%%%%%%%%%%

%\documentclass[runningheads]{llncs}
% Suppress page numbers
%\documentclass[a4paper]{llncs}
% For arXiv, and eprint support
\documentclass{article}

\usepackage{amssymb}
\usepackage{amsthm}
%\setcounter{tocdepth}{3}
\usepackage{graphicx}
\usepackage{hyperref}
\usepackage{eprint}
%\usepackage[osf]{mathpazo} % Use Palatino / Euler fonts
%\usepackage{multirow}
%\usepackage[retainorgcmds]{IEEEtrantools}

\usepackage{url}
\urldef{\mailsa}\path|{alfred.hofmann, ursula.barth, ingrid.beyer, christine.guenther,|
\urldef{\mailsb}\path|frank.holzwarth, anna.kramer, erika.siebert-cole, lncs}@springer.com|
\newcommand{\keywords}[1]{\par\addvspace\baselineskip
\noindent\keywordname\enspace\ignorespaces#1}

% Right brace for multirows in tables / arrays
\newcommand\coolrightbrace[2]{%
\left.\vphantom{\begin{matrix} #1 \end{matrix}}\right\}#2}

% To fix Qcircuit target with new Xypic
\newcommand{\targfix}{\qw {\xy {<0em,0em> \ar @{ - } +<.39em,0em>
\ar @{ - } -<.39em,0em> \ar @{ - } +
<0em,.39em> \ar @{ - }
-<0em,.39em>},<0em,0em>*{\rule{.01em}{.01em}}*+<.8em>\frm{o}
\endxy}}

% To get Roman uppercase Greek characters
\newcommand{\X}[1]{$#1$}

%    Q-circuit version 1.06
%    Copyright (C) 2004  Steve Flammia & Bryan Eastin

%    This program is free software; you can redistribute it and/or modify
%    it under the terms of the GNU General Public License as published by
%    the Free Software Foundation; either version 2 of the License, or
%    (at your option) any later version.
%
%    This program is distributed in the hope that it will be useful,
%    but WITHOUT ANY WARRANTY; without even the implied warranty of
%    MERCHANTABILITY or FITNESS FOR A PARTICULAR PURPOSE.  See the
%    GNU General Public License for more details.
%
%    You should have received a copy of the GNU General Public License
%    along with this program; if not, write to the Free Software
%    Foundation, Inc., 59 Temple Place, Suite 330, Boston, MA  02111-1307  USA

\usepackage[matrix,frame,arrow]{xy}
\usepackage{amsmath}
\newcommand{\bra}[1]{\left\langle{#1}\right\vert}
\newcommand{\ket}[1]{\left\vert{#1}\right\rangle}
    % Defines Dirac notation.
\newcommand{\qw}[1][-1]{\ar @{-} [0,#1]}
    % Defines a wire that connects horizontally.  By default it connects to the object on the left of the current object.
    % WARNING: Wire commands must appear after the gate in any given entry.
\newcommand{\qwx}[1][-1]{\ar @{-} [#1,0]}
    % Defines a wire that connects vertically.  By default it connects to the object above the current object.
    % WARNING: Wire commands must appear after the gate in any given entry.
\newcommand{\cw}[1][-1]{\ar @{=} [0,#1]}
    % Defines a classical wire that connects horizontally.  By default it connects to the object on the left of the current object.
    % WARNING: Wire commands must appear after the gate in any given entry.
\newcommand{\cwx}[1][-1]{\ar @{=} [#1,0]}
    % Defines a classical wire that connects vertically.  By default it connects to the object above the current object.
    % WARNING: Wire commands must appear after the gate in any given entry.
\newcommand{\gate}[1]{*{\xy *+<.6em>{#1};p\save+LU;+RU **\dir{-}\restore\save+RU;+RD **\dir{-}\restore\save+RD;+LD **\dir{-}\restore\POS+LD;+LU **\dir{-}\endxy} \qw}
    % Boxes the argument, making a gate.
\newcommand{\meter}{\gate{\xy *!<0em,1.1em>h\cir<1.1em>{ur_dr},!U-<0em,.4em>;p+<.5em,.9em> **h\dir{-} \POS <-.6em,.4em> *{},<.6em,-.4em> *{} \endxy}}
    % Inserts a measurement meter.
\newcommand{\measure}[1]{*+[F-:<.9em>]{#1} \qw}
    % Inserts a measurement bubble with user defined text.
\newcommand{\measuretab}[1]{*{\xy *+<.6em>{#1};p\save+LU;+RU **\dir{-}\restore\save+RU;+RD **\dir{-}\restore\save+RD;+LD **\dir{-}\restore\save+LD;+LC-<.5em,0em> **\dir{-} \restore\POS+LU;+LC-<.5em,0em> **\dir{-} \endxy} \qw}
    % Inserts a measurement tab with user defined text.
\newcommand{\measureD}[1]{*{\xy*+=+<.5em>{\vphantom{#1}}*\cir{r_l};p\save*!R{#1} \restore\save+UC;+UC-<.5em,0em>*!R{\hphantom{#1}}+L **\dir{-} \restore\save+DC;+DC-<.5em,0em>*!R{\hphantom{#1}}+L **\dir{-} \restore\POS+UC-<.5em,0em>*!R{\hphantom{#1}}+L;+DC-<.5em,0em>*!R{\hphantom{#1}}+L **\dir{-} \endxy} \qw}
    % Inserts a D-shaped measurement gate with user defined text.
\newcommand{\multimeasure}[2]{*+<1em,.9em>{\hphantom{#2}} \qw \POS[0,0].[#1,0];p !C *{#2},p \drop\frm<.9em>{-}}
    % Draws a multiple qubit measurement bubble starting at the current position and spanning #1 additional gates below.
    % #2 gives the label for the gate.
    % You must use an argument of the same width as #2 in \ghost for the wires to connect properly on the lower lines.
\newcommand{\multimeasureD}[2]{*+<1em,.9em>{\hphantom{#2}}\save[0,0].[#1,0];p\save !C *{#2},p+LU+<0em,0em>;+RU+<-.8em,0em> **\dir{-}\restore\save +LD;+LU **\dir{-}\restore\save +LD;+RD-<.8em,0em> **\dir{-} \restore\save +RD+<0em,.8em>;+RU-<0em,.8em> **\dir{-} \restore \POS !UR*!UR{\cir<.9em>{r_d}};!DR*!DR{\cir<.9em>{d_l}}\restore \qw}
    % Draws a multiple qubit D-shaped measurement gate starting at the current position and spanning #1 additional gates below.
    % #2 gives the label for the gate.
    % You must use an argument of the same width as #2 in \ghost for the wires to connect properly on the lower lines.
\newcommand{\control}{*-=-{\bullet}}
    % Inserts an unconnected control.
\newcommand{\controlo}{*!<0em,.04em>-<.07em,.11em>{\xy *=<.45em>[o][F]{}\endxy}}
    % Inserts a unconnected control-on-0.
\newcommand{\ctrl}[1]{\control \qwx[#1] \qw}
    % Inserts a control and connects it to the object #1 wires below.
\newcommand{\ctrlo}[1]{\controlo \qwx[#1] \qw}
    % Inserts a control-on-0 and connects it to the object #1 wires below.
\newcommand{\targ}{*{\xy{<0em,0em>*{} \ar @{ - } +<.4em,0em> \ar @{ - } -<.4em,0em> \ar @{ - } +<0em,.4em> \ar @{ - } -<0em,.4em>},*+<.8em>\frm{o}\endxy} \qw}
    % Inserts a CNOT target.
\newcommand{\qswap}{*=<0em>{\times} \qw}
    % Inserts half a swap gate. 
    % Must be connected to the other swap with \qwx.
\newcommand{\multigate}[2]{*+<1em,.9em>{\hphantom{#2}} \qw \POS[0,0].[#1,0];p !C *{#2},p \save+LU;+RU **\dir{-}\restore\save+RU;+RD **\dir{-}\restore\save+RD;+LD **\dir{-}\restore\save+LD;+LU **\dir{-}\restore}
    % Draws a multiple qubit gate starting at the current position and spanning #1 additional gates below.
    % #2 gives the label for the gate.
    % You must use an argument of the same width as #2 in \ghost for the wires to connect properly on the lower lines.
\newcommand{\ghost}[1]{*+<1em,.9em>{\hphantom{#1}} \qw}
    % Leaves space for \multigate on wires other than the one on which \multigate appears.  Without this command wires will cross your gate.
    % #1 should match the second argument in the corresponding \multigate. 
\newcommand{\push}[1]{*{#1}}
    % Inserts #1, overriding the default that causes entries to have zero size.  This command takes the place of a gate.
    % Like a gate, it must precede any wire commands.
    % \push is useful for forcing columns apart.
    % NOTE: It might be useful to know that a gate is about 1.3 times the height of its contents.  I.e. \gate{M} is 1.3em tall.
    % WARNING: \push must appear before any wire commands and may not appear in an entry with a gate or label.
\newcommand{\gategroup}[6]{\POS"#1,#2"."#3,#2"."#1,#4"."#3,#4"!C*+<#5>\frm{#6}}
    % Constructs a box or bracket enclosing the square block spanning rows #1-#3 and columns=#2-#4.
    % The block is given a margin #5/2, so #5 should be a valid length.
    % #6 can take the following arguments -- or . or _\} or ^\} or \{ or \} or _) or ^) or ( or ) where the first two options yield dashed and
    % dotted boxes respectively, and the last eight options yield bottom, top, left, and right braces of the curly or normal variety.
    % \gategroup can appear at the end of any gate entry, but it's good form to pick one of the corner gates.
    % BUG: \gategroup uses the four corner gates to determine the size of the bounding box.  Other gates may stick out of that box.  See \prop. 
\newcommand{\rstick}[1]{*!L!<-.5em,0em>=<0em>{#1}}
    % Centers the left side of #1 in the cell.  Intended for lining up wire labels.  Note that non-gates have default size zero.
\newcommand{\lstick}[1]{*!R!<.5em,0em>=<0em>{#1}}
    % Centers the right side of #1 in the cell.  Intended for lining up wire labels.  Note that non-gates have default size zero.
\newcommand{\ustick}[1]{*!D!<0em,-.5em>=<0em>{#1}}
    % Centers the bottom of #1 in the cell.  Intended for lining up wire labels.  Note that non-gates have default size zero.
\newcommand{\dstick}[1]{*!U!<0em,.5em>=<0em>{#1}}
    % Centers the top of #1 in the cell.  Intended for lining up wire labels.  Note that non-gates have default size zero.
\newcommand{\Qcircuit}{\xymatrix @*=<0em>}
    % Defines \Qcircuit as an \xymatrix with entries of default size 0em.


\newcommand{\braket}[2]{\langle #1|#2 \rangle}
\newcommand{\normtwo}{\frac{1}{\sqrt{2}}}
\newcommand{\norm}[1]{\parallel #1 \parallel}
\newcommand{\email}[1]{\href{mailto:#1}{#1}}
\theoremstyle{plain} \newtheorem{lemma}{Lemma}

\begin{document}

%\mainmatter  % start of an individual contribution

% first the title is needed
\title{A 2D Nearest-Neighbor Quantum Architecture for Factoring}

% a short form should be given in case it is too long for the running head
%\titlerunning{A 2D Nearest-Neighbor Quantum Architecture for Factoring}

% the name(s) of the author(s) follow(s) next
%
% NB: Chinese authors should write their first names(s) in front of
% their surnames. This ensures that the names appear correctly in
% the running heads and the author index.
%
\author{Paul Pham\\
University of Washington\\
Quantum Theory Group\\
Box 352350, Seattle, WA 98195, USA,\\
\email{ppham@cs.washington.edu},\\
\url{http://www.cs.washington.edu/homes/ppham/}
\and
Krysta M. Svore\\
Microsoft Research\\
Quantum Architectures and Computation Group\\
One Microsoft Way, Redmond, WA 98052, USA\\
\email{ksvore@microsoft.com},\\
\url{http://research.microsoft.com/en-us/people/ksvore/}
}
% if the list of authors exceeds the space for a headline,
% an abbreviated author list is needed
%\authorrunning{P. Pham \and K.M. Svore}
% (feature abused for this document to repeat the title also on left hand pages)

\maketitle

This document responds to comments by Referee 2, which were received on
November 30, 2012. These comments are quoted and responded to below.
We thank the Referee for the constructive feedback and suggestions.

\section{General Comments}

\begin{quote}
The paper "A 2D Nearest-Neighbor Quantum Architecture for Factoring",
by Pham and Svore, contains new ideas and represents a step forward
in our understanding of how to implement arithmetic on a quantum
computer.  The authors do two things that separate their paper from
previous analyses of Shor's algorithm: they consider a 2D network,
and they choose to optimize depth rather than width.  While it is
still too early to say which physical constraints on quantum computers
will be most restrictive, the authors explore an important new part
of the space of algorithms.  This paper is appropriate for publication in QIC.

There are a few issues that should be addressed before publication.
\end{quote}

%\subsection{The introduction does not state the main result}

\begin{quote}
The introduction does not state the main result.  At the very
least, there should be a statement of the asymptotic behavior of
the circuit.  It is even possible that most (or all) of Section 8
should be moved earlier, including figure 11.  The reader should not
have to read to the end to find out the punch line.
\end{quote}

{\it We have reworded the introduction to reflect the main contribution of our work.}

%\subsection{The introduction to Section 4 is confusing.}

\begin{quote}
The introduction to Section 4 is confusing.  Is the pair $(u,v)$ a
CSE number only when it arises from this construction?  Is $u_{n-1} = 0 $
a convention within the paper or part of the definition?  The example
in Figure 2 might suggest that $u_i = v_i = 1$ is not permitted, when
in fact it is.  The authors need to clarify standard definitions versus
their conventions. A reference to pre-quantum literature on carry-save
addition would help.
\end{quote}

{\it  The sum of any two numbers (u+v) can be interpreted as a CSE number.
The point of the definition a+b+c = u+v is to associate a given u+v with a particular a+b+c,
but this association is non-unique. However, given the truth table for parity (u) and majority (v),
this is a one-to-one mapping between a,b,c and u,v. By that truth table, $u_{n-1} = 0$. }

{\it In the related work section, we have included a reference to Wallace Trees and the 3-2 adder.  These are the classical CSA techniques.}

%\subsection{Non-unique representations of the answer}

\begin{quote}
The authors imply that the final output of their exponentiation circuit
is left in CSE form, which is not a unique representation of the answer.
This could mess up the next step of Shor's algorithm, in which states with
the same answer collapse.  The authors need to either (a) explain why this
is not a problem, or (b) explicitly say that they convert the answer to
standard form.  (Note that, at worst, this conversion can be done in log
depth, so the asymptotic analysis of the algorithm is be affected.)
\end{quote}

{\it Thank you for catching this mistake. Resources have been added at the
end to upper bound the conversion of the final CSE output to a conventional
number.}

\section{Minor comments}

\begin{quote}
Page 1: "best-known" should be "best known".  ("best-known" means
"most famous", which is not what is intended.)
\end{quote}

{\it Corrected.}

\begin{quote}
Top of Section 2.1: Should be "following Van Meter and Itoh [Van Meter
and Itoh (2005)]" or "following [Van Meter and Itoh (2005)]".
\end{quote}

{\it Corrected.}

\begin{quote}
Page 2: ``where each qubit has four neighbors'' is misleading since it
implies a torus rather than a bounded planar region, and since the authors
are about to change four to six.  Simply ``where there is an extra...''
would suffice.
\end{quote}

{\it Corrected to ``planar" neighbors.}

\begin{quote}
Last sentence of Section 2: "there is no known way".  Do the authors
mean (a) no one has found a way, (b) there is provably no way, or
(c) there is provably no way for a particular family of circuits? The
at-large citation to Rosenbaum is unhelpful.  One way to clarify would be
to cite a specific result from Rosenbaum.
\end{quote}

{\it This sentence has been removed.  Recent private communication indicates a
constructive way to unentangle the source qubit.}

\begin{quote}
Bottom of page 6: "The circuit operations out-of-place and produces
two garbage qubits".  No.  An out-of-place circuit leaves its input intact;
this circuit overwrites $a_i$, so it operates in place.  Calling $b_i$ and
$c_i$ "garbage" seems not quite right, since if they were not present the
circuit would not be reversible.  One could design an out-of-place version,
but in this case cleaning up $b_i$ and $c_i$ would be straightforward and
there would be no garbage.
\end{quote}

{\it Sentence has been corrected.}

\begin{quote}
Page 7: The phrase "n-bit" appears twice in one sentence, once as a
an adjective and once as a noun, meaning two different things. This is
not technically ambiguous, but it is awkward.
\end{quote}

{\it Corrected.}

\begin{quote}
Page 7, before (6): The term "signficance" has not been defined and
seems not be to used elsewhere.  The sentence could be rewritten.
\end{quote}

{\it We added the definition of significance in Section 4.}

\begin{quote}
Page 7, proof of lemma 1: extraneous "." in "Our".
\end{quote}

{\it Corrected.}

\begin{quote}
Pages 8 and 9:  The argument that $v'''_{n+2} = 0$ is convoluted, working
backward from the conclusion to a true statement (and misusing the word
"implies" in the process), and contains at least one mistake.  It would be
cleaner to work forward.  For example: since $u'_{(n)} + v'_{(n+1)} =
u_{(n)} + v_{(n)} <= 2^{n+1}$, the bits $u'_n and v'_{n+1}$ cannot both be $1$.
But $u''_{n+1} = v'_{n+1}$ and $v''_{n+1} = u'_n v'_n$, so $u''_{n+1}$ and
$v''_{n+1}$ cannot both be $1$, and hence $v'''_{n+2} = 0$.
\end{quote}

{\it We thank the referee for his shorter proof and for pointing out the
mistake. We have incorporated this change into the paper.}

\bibliography{factor2d}
\bibliographystyle{tocplain}

\end{document}
