%%%%%%%%%%%%%%%%%%%%%%%%%%%%%%%%%%%%%%%%%%%%%%%%%%%%%%%%%%%%%%%%%%%%%%%%%%%%%%
% This is the LaTeX source for
% "A 2D Nearest-Neighbor Quantum Architecture for Factoring"
% submitted to the Reversible Computing Workshop (RC 2012)
% based on Spring-Verlag's Lecture Notes in Computer Sciences template
% typeinst.tex
% Paul Pham and Krysta Svore
% 14 March 2012
%%%%%%%%%%%%%%%%%%%%%%%%%%%%%%%%%%%%%%%%%%%%%%%%%%%%%%%%%%%%%%%%%%%%%%%%%%%%%%

%\documentclass[runningheads]{llncs}
% Suppress page numbers
%\documentclass[a4paper]{llncs}
% For arXiv, and eprint support
\documentclass{article}

\usepackage{amssymb}
\usepackage{amsthm}
%\setcounter{tocdepth}{3}
\usepackage{graphicx}
\usepackage{hyperref}
\usepackage{eprint}
%\usepackage[osf]{mathpazo} % Use Palatino / Euler fonts
%\usepackage{multirow}
%\usepackage[retainorgcmds]{IEEEtrantools}

\usepackage{url}
\urldef{\mailsa}\path|{alfred.hofmann, ursula.barth, ingrid.beyer, christine.guenther,|
\urldef{\mailsb}\path|frank.holzwarth, anna.kramer, erika.siebert-cole, lncs}@springer.com|
\newcommand{\keywords}[1]{\par\addvspace\baselineskip
\noindent\keywordname\enspace\ignorespaces#1}

% Right brace for multirows in tables / arrays
\newcommand\coolrightbrace[2]{%
\left.\vphantom{\begin{matrix} #1 \end{matrix}}\right\}#2}

% To fix Qcircuit target with new Xypic
\newcommand{\targfix}{\qw {\xy {<0em,0em> \ar @{ - } +<.39em,0em>
\ar @{ - } -<.39em,0em> \ar @{ - } +
<0em,.39em> \ar @{ - }
-<0em,.39em>},<0em,0em>*{\rule{.01em}{.01em}}*+<.8em>\frm{o}
\endxy}}

% To get Roman uppercase Greek characters
\newcommand{\X}[1]{$#1$}

%    Q-circuit version 1.06
%    Copyright (C) 2004  Steve Flammia & Bryan Eastin

%    This program is free software; you can redistribute it and/or modify
%    it under the terms of the GNU General Public License as published by
%    the Free Software Foundation; either version 2 of the License, or
%    (at your option) any later version.
%
%    This program is distributed in the hope that it will be useful,
%    but WITHOUT ANY WARRANTY; without even the implied warranty of
%    MERCHANTABILITY or FITNESS FOR A PARTICULAR PURPOSE.  See the
%    GNU General Public License for more details.
%
%    You should have received a copy of the GNU General Public License
%    along with this program; if not, write to the Free Software
%    Foundation, Inc., 59 Temple Place, Suite 330, Boston, MA  02111-1307  USA

\usepackage[matrix,frame,arrow]{xy}
\usepackage{amsmath}
\newcommand{\bra}[1]{\left\langle{#1}\right\vert}
\newcommand{\ket}[1]{\left\vert{#1}\right\rangle}
    % Defines Dirac notation.
\newcommand{\qw}[1][-1]{\ar @{-} [0,#1]}
    % Defines a wire that connects horizontally.  By default it connects to the object on the left of the current object.
    % WARNING: Wire commands must appear after the gate in any given entry.
\newcommand{\qwx}[1][-1]{\ar @{-} [#1,0]}
    % Defines a wire that connects vertically.  By default it connects to the object above the current object.
    % WARNING: Wire commands must appear after the gate in any given entry.
\newcommand{\cw}[1][-1]{\ar @{=} [0,#1]}
    % Defines a classical wire that connects horizontally.  By default it connects to the object on the left of the current object.
    % WARNING: Wire commands must appear after the gate in any given entry.
\newcommand{\cwx}[1][-1]{\ar @{=} [#1,0]}
    % Defines a classical wire that connects vertically.  By default it connects to the object above the current object.
    % WARNING: Wire commands must appear after the gate in any given entry.
\newcommand{\gate}[1]{*{\xy *+<.6em>{#1};p\save+LU;+RU **\dir{-}\restore\save+RU;+RD **\dir{-}\restore\save+RD;+LD **\dir{-}\restore\POS+LD;+LU **\dir{-}\endxy} \qw}
    % Boxes the argument, making a gate.
\newcommand{\meter}{\gate{\xy *!<0em,1.1em>h\cir<1.1em>{ur_dr},!U-<0em,.4em>;p+<.5em,.9em> **h\dir{-} \POS <-.6em,.4em> *{},<.6em,-.4em> *{} \endxy}}
    % Inserts a measurement meter.
\newcommand{\measure}[1]{*+[F-:<.9em>]{#1} \qw}
    % Inserts a measurement bubble with user defined text.
\newcommand{\measuretab}[1]{*{\xy *+<.6em>{#1};p\save+LU;+RU **\dir{-}\restore\save+RU;+RD **\dir{-}\restore\save+RD;+LD **\dir{-}\restore\save+LD;+LC-<.5em,0em> **\dir{-} \restore\POS+LU;+LC-<.5em,0em> **\dir{-} \endxy} \qw}
    % Inserts a measurement tab with user defined text.
\newcommand{\measureD}[1]{*{\xy*+=+<.5em>{\vphantom{#1}}*\cir{r_l};p\save*!R{#1} \restore\save+UC;+UC-<.5em,0em>*!R{\hphantom{#1}}+L **\dir{-} \restore\save+DC;+DC-<.5em,0em>*!R{\hphantom{#1}}+L **\dir{-} \restore\POS+UC-<.5em,0em>*!R{\hphantom{#1}}+L;+DC-<.5em,0em>*!R{\hphantom{#1}}+L **\dir{-} \endxy} \qw}
    % Inserts a D-shaped measurement gate with user defined text.
\newcommand{\multimeasure}[2]{*+<1em,.9em>{\hphantom{#2}} \qw \POS[0,0].[#1,0];p !C *{#2},p \drop\frm<.9em>{-}}
    % Draws a multiple qubit measurement bubble starting at the current position and spanning #1 additional gates below.
    % #2 gives the label for the gate.
    % You must use an argument of the same width as #2 in \ghost for the wires to connect properly on the lower lines.
\newcommand{\multimeasureD}[2]{*+<1em,.9em>{\hphantom{#2}}\save[0,0].[#1,0];p\save !C *{#2},p+LU+<0em,0em>;+RU+<-.8em,0em> **\dir{-}\restore\save +LD;+LU **\dir{-}\restore\save +LD;+RD-<.8em,0em> **\dir{-} \restore\save +RD+<0em,.8em>;+RU-<0em,.8em> **\dir{-} \restore \POS !UR*!UR{\cir<.9em>{r_d}};!DR*!DR{\cir<.9em>{d_l}}\restore \qw}
    % Draws a multiple qubit D-shaped measurement gate starting at the current position and spanning #1 additional gates below.
    % #2 gives the label for the gate.
    % You must use an argument of the same width as #2 in \ghost for the wires to connect properly on the lower lines.
\newcommand{\control}{*-=-{\bullet}}
    % Inserts an unconnected control.
\newcommand{\controlo}{*!<0em,.04em>-<.07em,.11em>{\xy *=<.45em>[o][F]{}\endxy}}
    % Inserts a unconnected control-on-0.
\newcommand{\ctrl}[1]{\control \qwx[#1] \qw}
    % Inserts a control and connects it to the object #1 wires below.
\newcommand{\ctrlo}[1]{\controlo \qwx[#1] \qw}
    % Inserts a control-on-0 and connects it to the object #1 wires below.
\newcommand{\targ}{*{\xy{<0em,0em>*{} \ar @{ - } +<.4em,0em> \ar @{ - } -<.4em,0em> \ar @{ - } +<0em,.4em> \ar @{ - } -<0em,.4em>},*+<.8em>\frm{o}\endxy} \qw}
    % Inserts a CNOT target.
\newcommand{\qswap}{*=<0em>{\times} \qw}
    % Inserts half a swap gate. 
    % Must be connected to the other swap with \qwx.
\newcommand{\multigate}[2]{*+<1em,.9em>{\hphantom{#2}} \qw \POS[0,0].[#1,0];p !C *{#2},p \save+LU;+RU **\dir{-}\restore\save+RU;+RD **\dir{-}\restore\save+RD;+LD **\dir{-}\restore\save+LD;+LU **\dir{-}\restore}
    % Draws a multiple qubit gate starting at the current position and spanning #1 additional gates below.
    % #2 gives the label for the gate.
    % You must use an argument of the same width as #2 in \ghost for the wires to connect properly on the lower lines.
\newcommand{\ghost}[1]{*+<1em,.9em>{\hphantom{#1}} \qw}
    % Leaves space for \multigate on wires other than the one on which \multigate appears.  Without this command wires will cross your gate.
    % #1 should match the second argument in the corresponding \multigate. 
\newcommand{\push}[1]{*{#1}}
    % Inserts #1, overriding the default that causes entries to have zero size.  This command takes the place of a gate.
    % Like a gate, it must precede any wire commands.
    % \push is useful for forcing columns apart.
    % NOTE: It might be useful to know that a gate is about 1.3 times the height of its contents.  I.e. \gate{M} is 1.3em tall.
    % WARNING: \push must appear before any wire commands and may not appear in an entry with a gate or label.
\newcommand{\gategroup}[6]{\POS"#1,#2"."#3,#2"."#1,#4"."#3,#4"!C*+<#5>\frm{#6}}
    % Constructs a box or bracket enclosing the square block spanning rows #1-#3 and columns=#2-#4.
    % The block is given a margin #5/2, so #5 should be a valid length.
    % #6 can take the following arguments -- or . or _\} or ^\} or \{ or \} or _) or ^) or ( or ) where the first two options yield dashed and
    % dotted boxes respectively, and the last eight options yield bottom, top, left, and right braces of the curly or normal variety.
    % \gategroup can appear at the end of any gate entry, but it's good form to pick one of the corner gates.
    % BUG: \gategroup uses the four corner gates to determine the size of the bounding box.  Other gates may stick out of that box.  See \prop. 
\newcommand{\rstick}[1]{*!L!<-.5em,0em>=<0em>{#1}}
    % Centers the left side of #1 in the cell.  Intended for lining up wire labels.  Note that non-gates have default size zero.
\newcommand{\lstick}[1]{*!R!<.5em,0em>=<0em>{#1}}
    % Centers the right side of #1 in the cell.  Intended for lining up wire labels.  Note that non-gates have default size zero.
\newcommand{\ustick}[1]{*!D!<0em,-.5em>=<0em>{#1}}
    % Centers the bottom of #1 in the cell.  Intended for lining up wire labels.  Note that non-gates have default size zero.
\newcommand{\dstick}[1]{*!U!<0em,.5em>=<0em>{#1}}
    % Centers the top of #1 in the cell.  Intended for lining up wire labels.  Note that non-gates have default size zero.
\newcommand{\Qcircuit}{\xymatrix @*=<0em>}
    % Defines \Qcircuit as an \xymatrix with entries of default size 0em.


\newcommand{\braket}[2]{\langle #1|#2 \rangle}
\newcommand{\normtwo}{\frac{1}{\sqrt{2}}}
\newcommand{\norm}[1]{\parallel #1 \parallel}
\newcommand{\email}[1]{\href{mailto:#1}{#1}}
\theoremstyle{plain} \newtheorem{lemma}{Lemma}

\begin{document}

%\mainmatter  % start of an individual contribution

% first the title is needed
\title{A 2D Nearest-Neighbor Quantum Architecture for Factoring}

% a short form should be given in case it is too long for the running head
%\titlerunning{A 2D Nearest-Neighbor Quantum Architecture for Factoring}

% the name(s) of the author(s) follow(s) next
%
% NB: Chinese authors should write their first names(s) in front of
% their surnames. This ensures that the names appear correctly in
% the running heads and the author index.
%
\author{Paul Pham\\
University of Washington\\
Quantum Theory Group\\
Box 352350, Seattle, WA 98195, USA,\\
\email{ppham@cs.washington.edu},\\
\url{http://www.cs.washington.edu/homes/ppham/}
\and
Krysta M. Svore\\
Microsoft Research\\
Quantum Architectures and Computation Group\\
One Microsoft Way, Redmond, WA 98052, USA\\
\email{ksvore@microsoft.com},\\
\url{http://research.microsoft.com/en-us/people/ksvore/}
}
% if the list of authors exceeds the space for a headline,
% an abbreviated author list is needed
%\authorrunning{P. Pham \and K.M. Svore}
% (feature abused for this document to repeat the title also on left hand pages)

\maketitle

This document responds to comments by Referee 1, which were received on
November 30, 2012. These comments are quoted and responded to below.  We thank the Referee for the constructive feedback and suggestions.

\section{General Comments}

\begin{quote}
In this paper, the authors present in detail a new version of the
modular exponentiation component of Shor's algorithm, with attention
to the constraints of a 2-D planar graph of moderate degree for qubit
connectivity.  They use teleportation-based fanout to move qubits
around within the machine.  The modular addition uses a novel method,
depending upon carry-save arithmetic and a small triangular lattice as
its unit cell.

Perhaps the most novel part of the arithmetic is the approach to
modular multiplication.  In a traditional modular multiplication of
two numbers $x$ and $y$, the $n^2$ bit-wise products $x_i y_j$ are calculated
and laid out in a trapezoid shifted to give the correct column (power
of two), such that each entry is $2^i 2^j x_i y_j$, which of course can
be represented by a single bit in the right place.  Then the columns
are added, creating a 2n-bit number, which then must be further
operated on to perform the modulo $N$ operation.  ($N$ is assumed to be an
$n$-bit number with the high-order bit being a one.)

In contrast, in this approach, each of the $n^2$ entries is a full $n$-bit
number, $(2^i 2^j \bmod m) x_i y_j$.  By adding those numbers using their
modular circuit, the full $x y \bmod N$ value can be calculated directly.
As proposed, this requires $O(n^3)$ bits (qubits) in the register.  By
combining the $n^2$ partial results in a log-depth tree structure, the
depth for a modular multiplication becomes $O(\log n)$ times the constant
depth of their 3-to-2 carry-save modular operation.

Up through section 5, I was prepared to recommend nearly immediate
acceptance of the paper.  The writing is clear and elegant and the
technical work both valuable and polished.  In sections 6 and 7,
however, I have some doubts about the technical work, and the writing
seems a bit more rushed, and the tail of the paper is not yet
satisfactory.

Most importantly, it is disappointing that the authors have not
produced a more complete estimate of the number of qubits (resources)
required, as well as the actual circuit depth.  A rough estimate, at
least, should be very achievable given the level of detail already
developed in the paper.  The authors mention this as future work, but
without it, the value of the paper is substantially diminished, and it
does not seem unreasonable to expect it to appear here.

\end{quote}

{\it We thank the Referee for the valuable feedback. 
We have included more detailed estimates of the resources required and improved
the writing in the latter sections of the paper.}

\section{Other significant technical comments:}

\begin{quote}
The authors appear not to have noticed that half of the $n^2$ numbers in
their multiplier require only a single bit.  As long as $i+j \le n$, the
modulo operation results in the same number, allowing us to avoid
using a full $n$-bit number: $(2^i 2^j \bmod m) x_i y_j = 2^i 2^j x_i y_j$.
\end{quote}

{\it Thank you for pointing this out.}

\section{Section 2.2}

\begin{quote}
``this 'consumes' the cat state'' -- this sentence is confusing.
\end{quote}

{\it We meant that the cat state remains entangled with the original source
qubit and its fanned-out, entangled copies. We have since discovered
via personal conversation with Aram Harrow and Dan Browne that this
statement is no longer true. It is possible to ``un-fanout'' and therefore
disentangle the cat state from the source and target qubits, allowing us
to reuse this state.  We have thus removed this sentence.}

\section{Section 3}

\begin{quote}
  Choi and Van Meter were not the first to consider 2-D architectures;
most of the solid-state proposals and even some of the ion trap
proposals worked with a 2-D layout.  Kielpinski's 2002 proposal might
have been the first 2-D architecture.  Working out exact algorithms on
the structures came later, but papers by Kubiatowicz's group and
Chong/Oskin clearly included at least some level of work on the
movement of qubits in a planar system, albeit with less attention to
the abstraction of logical qubit connectivity.
\end{quote}

{\it 
%We've added a reference to the 2002 Nature paper by Kielpinski et al.
%We have also included references to several quantum architecture papers in this section.
We intended to say that Choi and Van Meter were the first to construct an adder explicitly optimized for a 2D architecture of qubits.
We have reworded this paragraph to accurately reflect their contribution and added appropriate references.}

%KRYSTA TODO: Can you add references to quantum architecture papers by
%Kubi/Chong/Oskin on 2D implementations, since you are probably more
%familiar with those works?

\begin{quote}
  I'm not sure a modular adder is "extended" to do modular
exponentiation.  "Composed" or "used", perhaps?
\end{quote}

{\it We have reworded this sentence.}

\begin{quote}
  Your citation of Gossett has no year.
\end{quote}

{\it Corrected.}

\begin{quote}
"all other factoring implementations"?  That's a rather broad
characterization.  Cleve and Watrous long ago proposed using a
parallel reduction tree of multiplications before the QFT.  Van Meter
and Itoh investigated in detail the tradeoffs in resource consumption
for doing this.
\end{quote}

{\it We have clarified this sentence by referring to all nearest-neighbor factoring
implementations. By ``implementations,'' we mean a concrete mapping to an
architecture, such as those given by [Fowler et al. 2004] and [Kutin 2006].
While we do acknowledge the Cleve and Watrous paper, which gives
similar parallel results to those in the Kitaev-Shen-Vyalyi book, both
assume arbitrary connectivity of qubits.}

\begin{quote}
As far as I am aware, no one has worked out the details of Draper's
transform adder taking into account the need to do Solovay-Kitaev
decomposition.  This may add a very large factor to the execution
time.
\end{quote}

{\it Agreed, we are not aware of anyone working out compilation of the Draper
transform adder to a universal gate set, such as the Clifford group plus
$\pi/8$ gates. It's been shown that compilation will require $O(\log (1/\epsilon))$ overhead, where $\epsilon$ is the required precision. 
Throughout, we choose to avoid use of the QFT in our circuit due to the compilation required for each small rotation gate.  We have added a sentence regarding this compilation cost.
%how to compile more efficiently
%than Solovay-Kitaev using more time and space, but it would still not be
%more efficient than the Gossett adder.
}

\begin{quote}
Do you think the Zalka approximate multiplier approach actually works?
\end{quote}

{\it It is likely that something similar to the Zalka approximate
multiplier actually works in practice for the majority of input values,
if not the exact implementation described in
the Zalka and Kutin papers. However, a rigorous theoretical argument, or
empirical verification by simulation, has not yet appeared in the literature.
}

\begin{quote}
While it's okay to include a "forward reference" to a forthcoming
paper of your own that carries more detail, you can't ask us to "refer
to" it!
\end{quote}

{\it Agreed. Citation has been removed.}

\begin{quote}
I don't think the BKP ``exact" circuit guarantees a result, does it?
Shor's own original algorithm only probabilistically gives the correct
answer, and that probability is a matter of some debate in the
literature.  Papers by Fowler (2004), Miquel (1996), Garcia-Mata
(2007) and others provide different estimates.
\end{quote}

{\it Correct.  In this work we only analyze the resource cost for a single round of QPF.  The probability of achieving a correct answer may vary.}

\begin{quote}
The journal style will ultimately dictate this, but when the
bibliographic labels in the text are alphabetic, the references are
usually ordered alphabetically.
\end{quote}

{\it We have updated the style.}

%%%%%%%%%%%%%%%%%%%%%%%%%%%%%%%%%%%%%%%%%%%%%%%%%%%%%%%%%%%%%%%%%%%%%%%%%%%%%%%
\section{Section 4}

\begin{quote}
It would be worth pointing out that qubit 0 is the low-order qubit.
\end{quote}

{\it Corrected.}

\begin{quote}
``At the level of bits, a CSA...'' this sentence is awkward, reword.
\end{quote}

{\it Corrected.}

\begin{quote}
Fig. 4 shows the layout, but it doesn't exactly match the circuit of
Fig. 3.  Having the exact circuit to accompany Fig. 4 would be useful.
\end{quote}

{\it The layout in Fig. 4 matches Fig. 3 if you look at the outgoing qubit values in Fig. 3.  We have clarified this in the text.}

%%%%%%%%%%%%%%%%%%%%%%%%%%%%%%%%%%%%%%%%%%%%%%%%%%%%%%%%%%%%%%%%%%%%%%%%%%%%%%%
\section{Section 5}

\begin{quote}
It feels somewhat like the phrasing on constant depth is a bit
misleading in this section.  Please reread and make sure it is easy
for the reader to follow your claims.
\end{quote}

PAUL TODO
{\it Add response.}

\begin{quote}
Proof of Lemma 1: "O.ur" typo.
\end{quote}

{\it Corrected.}

\begin{quote}
A little attention to classical versus quantum addends in this section
might help the reader.
\end{quote}

PAUL TODO
{\it Add response.}

\begin{quote}
Fig. 5: labeling the lines themselves as "Layer 1", "Layer 2",
etc. might help the reader.
\end{quote}

PAUL TODO
{\it Add response.}

\begin{quote}
"at no layer generates" --> "no layer generates"?
\end{quote}

{\it Corrected.}

\begin{quote}
Fig. 6: Swap the left and right ends of this figure to make it
correspond to Fig. 5 as closely as possible, and point out this
correspondence by also labeling the "layers" and the time axis here.
"FANOUT RAIL" -- of which variable(s)?
\end{quote}

PAUL TODO
{\it Add response.}

\begin{quote}
My estimate here is that spatial resources are 29n qubits, temporal
resources are 12n Toffoli gates.  Is that about right?
\end{quote}

PAUL TODO
{\it Add response.}

%%%%%%%%%%%%%%%%%%%%%%%%%%%%%%%%%%%%%%%%%%%%%%%%%%%%%%%%%%%%%%%%%%%%%%%%%%%%%%%
\section{Section 6}

\begin{quote}
If you are worried that "quantum number" will mislead some readers,
would "quantum integer" be better?
\end{quote}

PAUL TODO
{\it Add response.}

\begin{quote}
Bottom of 6.1 is the place to expand the discussion to incorporate the
above on the size of partial products and their impact on resource
utilization.
\end{quote}

PAUL TODO
{\it Add response.}

\begin{quote}
Your white tiles are hard to see on my printout.
\end{quote}

PAUL TODO
{\it Add response.}

\begin{quote}
Important: Your second and third rules under "black tiles" appear to
conflict.  Reword to be precise.
\end{quote}

PAUL TODO
{\it Add response.}

\begin{quote}
Bottom of Sec. 6: more discussion of the size and number of addends
and resource consumption is needed.
\end{quote}

PAUL TODO
{\it Add response.}

\begin{quote}
Fig. 10: In earlier figures, arrows were used only to indicate addend
motion via teleportation, correct?  A different symbol to indicate
actual multiplication would make the figure clearer.  Otherwise,
asserting that an arrow itself is "log n depth" will confuse the
reader.  The tree approach used in this figure is not original, and in
fact dates to the 2000 paper by Cleve and Watrous, at least.
\end{quote}

{\it We have clarified the use of arrows in Fig. 10.}

{\it We do not claim that this tree structure is original, merely that it is
now possible to do the necessary communication (teleportation and fan-out)
in constant depth. That is, the tree structure before was a theoretical
construction, and now based on the previous nearest-neighbor implementations
of modular arithmetic, correspond to an actual physical tree structure if
this architecture were to be fabricated and observed to run over time.}

\begin{quote}
Sections 7, 8 and 9 should be extended with more discussion, and
actual resource consumption figures.  Some of the papers cited in
Fig. 11 (which ought to be a table, not a figure) provide detailed
estimates on resources (depth and width), as do Beckman et al. (1996,
uncited here, but should be) and Van Meter and Itoh.  Both of those
latter papers offer several configurations that may make direct
comparison tricky.
\end{quote}

PAUL TODO
{\it Add response.}

\begin{quote}
Overall, as noted, this paper will be a valuable contribution to the
literature once these issues are addressed.
\end{quote}

{\it We thank the referee for the valuable feedback.}

\bibliography{factor2d}
\bibliographystyle{tocplain}

\end{document}
