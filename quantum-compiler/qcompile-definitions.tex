\section{Preliminaries}
\label{sec:prelims}

\subsection{Some Special Operator Notation}

Borrowing the notation in \cite{ksv02},
we can define two "meta-operators" which takes some unitary $U$ as
a parameter. The first describes a controlled-$U$ operation where the control
is some single qubit.

\begin{displaymath}
\Lambda(U) = \ket{0}\bra{0} \otimes I + \ket{1}\bra{1} \otimes U
\end{displaymath}

The second describes a registered-$U$ operation, which
can be thought of as controlled on some multi-qubit register $\ket{p}$
encoding an $m$-qubit number $p$ to apply $U$ a certain number of times to a
target
register.

\begin{displaymath}
\Upsilon_m(U) : \ket{p} \otimes \ket{\psi} \rightarrow \ket{p}
\otimes U^p\ket{\psi}
\end{displaymath}

\subsection{A Universal Set of Gates}

We use the following universal standard set of gates $\mathcal{G}$.
The single-qubit rotations about the $x-$ and $z-$ axes
(of which $X$ and $Z$
are special cases) are known to be efficient on ion trap
implementations, but the set of realizable angles $\theta$ is finite.

\begin{displaymath}
\mathcal{G} = \{ H, K, K^{-1}, X, Z,
\Lambda(\sigma_x), \Lambda^2(\sigma_x) \}
\end{displaymath}

$\Lambda(\sigma_x)$ and $\Lambda^2(\sigma_x)$ are CNOT and Toffoli,
respectively.
Any current or future physical implementations of a quantum
computer will need to efficiently implement this set or an equivalent one.
Without proving the universality of $\mathcal{G}$, we note that all known
quantum algorithms reduce to it.

\subsection{Parameters}

The problem of quantum compiling is to translate
an entire circuit $C$ of $L$ gates with depth
$d$ to a new, compiled circuit $C'$ of size $L'$ and depth $d'$ which approximates
$C$ within some error $\epsilon$ using some distance measure.

In our code, we use the trace measure introduced by Austin Fowler which disregards
the global phase factor, so that we don't waste time trying to approximate
the unmeasurable phase of our target gate. Here,
$l$ refers to the dimensionality
of our system (for $n = 2^l$ qubits).

\begin{equation}
d(U,\tilde{U}) = \sqrt{\frac{l - \norm{\mathrm{tr}(U^\dagger \tilde{U})}}{l}}
\end{equation}

We will be somewhat sloppy and use the terms ``error'', ``precision'', and
``accuracy'' interchangeably when approximating gates.
There is some overhead in the compiled circuit, so in
general $C'$ is larger (that is, $L' > L$ and $d' > d$). It's also known that
in order to approximate a circuit with $L$ gates to a total precision of
$\epsilon$
requires each gate to be approximated to a precision of
$n = O(\log(L/\epsilon)$. We'll call the classical preprocessing time to
produce $C'$ as $T$.
 
Circuit depth is a heuristic for how parallelizable our
circuit is. For example, in an ion trap, if we had multiple lasers,
we could ``flatten'' our circuit into layers with bounded fan-in and
fan-out and operate on multiple ions in parallel.
All other things being equal, a circuit with low depth will complete
faster than one with high depth, although in practice we can only execute
fixed-width circuits.

\subsection{Quantum Coprocessor Model}

All experimental implementations of quantum computers treat them as an
auxiliary device controlled by a classical computer. This is the way
quantum computers will function for the foreseeable future, and many
quantum algorithms can actually be split into classical and quantum parts
to reflect this distinction. For example,
Solovay-Kitaev is a completely classical algorithm which is run before
a quantum algorithm to yield a deterministic set of gates. Super-Kitaev
also contains classical postprocessing as part of its parallelized
phase-estimation. Since classical computers are well understood and
pretty fast, we will neglect the performance of these classical parts
if they are polynomial in time. However, we will discuss one aspect
of classical overhead later since it appears to be intractable.

\subsection{Asymptotic Bounds}

The Solovay-Kitaev and Super-Kitaev algorithms compile circuits with
a size and depth which depend on the desired precision via the
parameter $n$ as shown below. We'll see these asymptotic bounds reflected
later in the actual numerical results in Section \ref{sec:results}.
The version of Solovay-Kitaev in Section \ref{sec:sk-algo} has a
larger exponent for the logarithm but is
easier to understand.

\begin{center}
\begin{table}
\begin{tabular}{|c|c|c|}
\hline
   & Solovay-Kitaev & Super-Kitaev\\
\hline
$L'$ & $O(Ln^{3+\nu})$ & $O(Ln + n^2 \log n)$\\
$d'$ & $O(dn^{3+\nu})$ & $O(d \log{n} + (\log{n})^2))$\\ 
\hline
\end{tabular}
\end{table}
\end{center}

where $\nu$ is a small positive constant.
The improve Solovay-Kitaev with the improved $3+\nu$ exponent and
Super-Kitaev are both described in \cite{ksv02}.