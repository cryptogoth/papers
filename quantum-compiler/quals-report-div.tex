\subsection{Quantum Divider}
\label{subsec:div}

The most resource-intensive step of the registered phase shift operation
is solving $p = p(s,l)$ in Equation
\ref{eqn:psl} for a given resulting
eigenvector $\ket{\psi_{2s-1}}$ after phase estimation and a desired
phase shift $exp(2\pi i l / 2^n)$.
This involves finding a modular inverse, which
we can convert to a product of sums (addition and multiplication,
which we already know how to do efficiently).

To find $y = (2s-1)^{-1} \mod 2^n$, it should have the property
$y \cdot (2s-1) \equiv 1 \mod 2^n$ or equivalently
$-y \cdot (2s-1) \equiv -1 \mod 2^n$. As a guess, we will choose the
$-y$ version and have it be a fraction with $(2s-1)$ in the denominator and
something in the numerator which is $1$ less than a multiple of
$2^n$, such as $2^n s^n - 1$.
We choose this to resemble the closed form of a geometric series,
as shown below.

\begin{displaymath}
-y = \frac{2^n s^n - 1}{2s - 1} = \frac{(2s)^n - 1}{(2s) - 1} =
\frac{1 - (2s)^n}{1 - (2s)} = \sum_{i=0}^n (2s)^i
\end{displaymath}

We use the following observation to convert this geometric series into a
product of sums involving power-of-two exponents, which makes for 
easy multiplication via bitshifting.

\begin{displaymath}
\sum_{k=0}^{2^l - 1} z^k = (1 + z)(1 + z^2)(1 + z^4)\cdots(1 + z^{2^{l-1}})
\end{displaymath}

In general a geometric series with $m = 2^l$ terms
involves $m$ sums
and $m$ products. Using the above formula, we have $\log m$ products and
$\log m$ sums. Both are necessary to achieve our desired logarithmic
depth for the overall registered phase shifting procedure.

Substituting, we get:

\begin{displaymath}
-y = \prod_{r=0}^{\lceil \log_n \rceil - 1} (1 + (2s)^{2^r})
\end{displaymath}

Then the final solution of $p$ in the first equation above is:

\begin{displaymath}
p = -yl \mod 2^n = -l \prod_{r=1}^{\lceil \log n \rceil -1} (1 + (2s)^{2^r})
\end{displaymath}

The resource counts for this modular division is equivalent to:

\begin{multline}
\text{DIV-MOD}(n) = \lceil \log n \rceil \cdot \text{MULT-2}(n) + \\
 (\lceil \log n \rceil - 1) \cdot \text{ADD}(2, n)
\end{multline}