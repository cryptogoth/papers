\section{Related Work}
\label{sec:related}

In 1995, Seth Lloyd found that almost any two distinct single-qubit rotations are
universal for approximating an arbitrary single-qubit rotation, but that this
approximation could be exponentially long in both time and length $T,L = (O(1/\epsilon))$ \cite{Lloyd1995}.

The theorem which is now called Solovay-Kitaev was discovered by Solovay in
1995 in an unpublished manuscript and independently later discovered by
Kitaev in 1997 \cite{nc00} which showed that $T,L = O(\log^c{1/\epsilon})$ for
$c$ between 3 and 4.

In 2001, Aram Harrow completed his undergrad thesis arguing that it would be difficult
to beat $c < 2$ for the above bounds using a successive approximation
method\cite{harrow01}.

In 2002, Kitaev, Shen, and Vyalyi published their book which contains an
application of parallelized phase estimation towards simulating a quantum
circuit (what we are calling Super-Kitaev) \cite{ksv02}.
That is, an alternative quantum compiler to Solovay-Kitaev which has
asymptotically better circuit depth and $T=O(1)$
but using ancillary qubits and increased
circuit size.

In 2003, Harrow, Recht, and Chuang demonstrated that a certain universal
set could be used to saturate the lower bound $L=O(\log{1/\epsilon})$
but it remains an open problem whether any efficient algorithm exists
which can do this in tractable $T$ \cite{hrc02}.

In 2005, Dawson and Nielsen published their pedagogical review
paper of Solovay-Kitaev \cite{Dawson2005}.

In 2010, Burrello, Mussardo, and Wan discovered a quantum compiler for
topological quantum computing which saturates the lower bound ($c=1$) for
non-Abelian anyons \cite{Burrello2010}.