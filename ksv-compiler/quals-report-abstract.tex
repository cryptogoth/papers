\section{Abstract}

Quantum compilers approximate high-level circuits for quantum algorithms
into a universal, machine-dependent
instruction set that can be implemented by experiment. This work contributes
a numerical comparison of the
resources needed to run two quantum compiling procedures on single-qubit gates,
along with the underlying open source code.
The first is the well-known Solovay-Kitaev procedure, which uses no ancilla and
is completely classical. The second is the lesser-known result by Kitaev,
Shen, and Vyalyi which reduces the compiled
circuit depth using parallelization at the expense of more ancilla qubits and
a quantum component that must be executed at runtime.
We also provide a pedagogical review of this second procedure as an aid to
future implementation and comment on some considerations of future
quantum architectures regarding compilation.