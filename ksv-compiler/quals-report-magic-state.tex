\subsection{Magic States}
\label{subsec:magic-state}

Suppose you had some magic state $\ket{\psi}$, given to you from some
all-powerful being, parameterized by the
integers $n$ and $k$ such that

\begin{equation*}
\ket{\psi_{n,k}} = \frac{1}{\sqrt{2^n}} \sum_{j=0}^{2^n-1}
e^{-2\pi i j k / 2^n} \ket{j}
\end{equation*}

where $0 < k < 2^n$, $k$ is odd.

Note that the magic states $\ket{\psi}$ look suspiciously like a
QFT state of the computational basis.
However that doesn't help us here, because QFT in general requires
$O(n^2)$-depth and we need a logarithmic-depth compiler.
This is intuitively why these states are hard to create.
If we were able to construct them easily, we would also have a quicker way
of implementing QFT.

However, what we can do is create a superposition over all odd $k=(2s-1)$
by starting in the state $\ket{0}^{\otimes n}$,
applying a Hadamard to the most significant qubit to get an equal mix of $\ket{0}$
and $\ket{1}$, then put a negative sign on the $\ket{1}$ component by applying
$\sigma^z$ to the same qubit.

\begin{equation*}
\ket{\eta} = \normtwo \ket{0} - \normtwo \ket{2^{n-1}} =
\frac{1}{\sqrt{2^{n-1}}} \sum_{s=1}^{2^{n-1}} \ket{\psi_{n,2s-1}}
\end{equation*}

More obviously relevant to our overall goal of approximating
$\Lambda(e^{i\phi})$, we can enact a phase
shift simply by doing addition, which is a result of these states
being eigenvectors of the modular addition operator defined
%by $A : \ket{j} \rightarrow \ket{(j+1) mod 2^n}$
 
\begin{equation*}
A\ket{\psi_{n,k}} = e^{2\pi i \phi_k} \ket{\psi_{n,k}}
\end{equation*}

\begin{equation*}
A\ket{j} \rightarrow \ket{j+1 \mod 2^n}
\end{equation*}

where finding the eigenvalue $e^{2\pi i \phi_k}$ corresponds to finding
the phase $\phi_k = k / 2^n$.

Repeated application of $A$ (say $p$ times) would result in a phase
added to the eigenstate equal to a multiple of $e^{2\pi i p / 2^n}$

\begin{equation}
A^p\ket{\psi_{n,k}} = e^{2\pi i \phi_k / 2^n} \ket{\psi_{n,k}}
\end{equation}

This explains why we don't find even $k$ interesting,
since then we would not get a
cyclic distribution of $2^n$ different phases,
since only odd $k$
are coprime with $2^n$. The exception is $k=0$, since this is the
equal superposition of computational basis states, which we can also
efficiently create. This will be a useful starting point later on to
create magic states
for odd $k$.

\begin{displaymath}
\ket{\psi_{n,0}} = H^{\otimes n}\ket{0^n}
\end{displaymath}

Suppose we have a certain state $\psi_{n,k}$ but we want to get enact
a phase shift $e^{2\pi i l / 2^n}$. We can do this by solving $p=p(s,l)$
in this equation:

\begin{equation}
\label{eqn:psl}
(2s-1)p \equiv l (\mod 2^n)
\end{equation}

Stipulating $k$ to be odd guarantees that there is a unique solution $p$.

We then applying $A^p$ as follows:

\begin{equation}
\label{eqn:upsilon}
\Upsilon_n(A) \ket{p, \psi_{n,k}} \rightarrow
e^{2\pi i l/2^n} \ket{p, \psi_{n,k}}
\end{equation}

But $p$ is in general hard to solve for, unless $k=1$, in which case
$p = l$. Therefore, we will later see that $\ket{\psi_{n,1}}$ is a desirable
state to have.

Finally, to copy the state $\ket{\psi_{n,k}}$ it suffices to apply the following
operator which only uses subtraction (addition with one addend and the
outcome negated in two's complement representation).

\begin{equation*}
\ket{\psi_{n,k}}^{\otimes m} = W^{-1}\left( \ket{\psi_{n,0}}^\otimes(m-1) \otimes \ket{\psi_{n,k}} \right)
\end{equation*}

where $W$ is defined by

\begin{multline}
W : \ket{x_1,\ldots,x_{m-1},x_m} \rightarrow \\
 \ket{x_1,\ldots,x_{m-1},x_1+\ldots+x_m}
\end{multline}

Armed with these properties, we're now ready to enact arbitrary phase shifts.