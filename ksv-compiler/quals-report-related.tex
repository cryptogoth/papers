\section{Related Work}
\label{sec:related}

In 1995, Seth Lloyd found that almost any two distinct single-qubit rotations are
universal for approximating an arbitrary single-qubit rotation, but that this
approximation could be exponentially long in the desired per-gate error $n$
for both time and length, that is,
$T,L = O(1/\epsilon)$ for $\epsilon = O(2^{-n})$ \cite{Lloyd1995}.

The theorem which is now called Solovay-Kitaev was discovered by Solovay in
1995 in an unpublished manuscript and independently later discovered by
Kitaev in 1997 \cite{nc00} which showed that $T,L = O(\log^c{1/\epsilon})$ for
$c$ between 3 and 4, showing for the first time that quantum compiling
could be efficient.

In 2001, Aram Harrow completed his undergrad thesis arguing that it would be difficult
to beat $c < 2$ for the above bounds using a successive approximation
method \cite{harrow01}.

In 2002, Kitaev, Shen, and Vyalyi published their book which contains an
application of parallelized phase estimation towards simulating a quantum
circuit (what we are calling KSV) \cite{ksv02}.
That is, KSV is an alternative quantum compiler to SK which has
asymptotically better circuit depth and circuit size, constant classical
preprocessing time ($T=O(1)$),
but using ancillary qubits (increased circuit width).

In 2003, Harrow, Recht, and Chuang demonstrated that a certain universal
set could be used to saturate the lower bound $L=O(\log{1/\epsilon})$
but it remains an open problem whether any efficient algorithm exists
which can do this in tractable $T$ \cite{hrc02}.

In 2005, Dawson and Nielsen published their pedagogical review
paper of Solovay-Kitaev \cite{Dawson2005}.

In 2010, Burrello, Mussardo, and Wan discovered a quantum compiler for
topological quantum computing which saturates the lower bound ($c=1$) for
non-Abelian anyons \cite{Burrello2010}.

%%%%%%%%%%%%%%%%%%%%%%%%%%%%%%%%%%%%%%%%%%%%%%%%%%%%%%%%%%%%%%%%%%%%%%%%%%%%%%
%%%%%%%%%%%%%%%%%%%%%%%%%%%%%%%%%%%%%%%%%%%%%%%%%%%%%%%%%%%%%%%%%%%%%%%%%%%%%%
% Write up references to Tucci and Markov and circuit synthesis papers
% put in the right chronological order as above