%%%%%%%%%%%%%%%%%%%%%%%%%%%%%%%%%%%%%%%%%%%%%%%%%%%%%%%%%%%%%%%%%%%%%%%%%%%%%%
%%%%%%%%%%%%%%%%%%%%%%%%%%%%%%%%%%%%%%%%%%%%%%%%%%%%%%%%%%%%%%%%%%%%%%%%%%%%%%
\section{Error Calculations}

In this section, we will calculate two error probabilities $\delta$ for
an approximate measuring operator $\tilde{Y}$, given that we are able to
approximate a function $f$ with error probability $\epsilon$. The first
error probability, which follows exactly the development in Problem 12.2
of Ref \cite{Kitaev2002}, assumes that we measure \emph{coherently}, meaning
that at no point do we projectively measure, and we are able to perfectly
uncompute all garbage. This is the ideal case, which we calculate in
\ref{subsec:error-noproj}. The second error probability, which we use in
\cite{Pham2012a}, involves projectively measuring as part of the operator $W$,
which involves some purely classical reversible operations that can be
offloaded to a classical controller. Afterwards, we execute $V$ as before,
but it is now impossible to run $W^{-1}$ because of the projective measurement.
This leaves some amount of garbage in the ancillary
subsystem $\mathcal{B}^N$, and we calculate it in \ref{subsec:error-proj}.

%%%%%%%%%%%%%%%%%%%%%%%%%%%%%%%%%%%%%%%%%%%%%%%%%%%%%%%%%%%%%%%%%%%%%%%%%%%%%%
%%%%%%%%%%%%%%%%%%%%%%%%%%%%%%%%%%%%%%%%%%%%%%%%%%%%%%%%%%%%%%%%%%%%%%%%%%%%%%
\subsection{Error Probability With Coherent Measurement}
\label{subsec:error-noproj}

First, we use this preliminary lemma. using properties of
the operator norm.

%%%%%%%%%%%%%%%%%%%%%%%%%%%%%%%%%%%%%%%%%%%%%%%%%%%%%%%%%%%%%%%%%%%%%%%%%%%%%%
\begin{lemma}[Solution 12.1, p. 230 \cite{ksv02}]
\label{lemma:sum-norm}
Let $X_j : \mathcal{N}_j \rightarrow \mathcal{M}_j $
be a collection of unitary operators which operate on pairwise orthogonal
subspaces, where each $X_j$ takes $\mathcal{N}_j$ to $\mathcal{M}_j$.
Then the norm of the operator $X$ formed as a direct product of these
$X_j$'s has an operator norm equal to the maximum of any of the
$X_j$'s.

\begin{equation}
X = \bigoplus_j X_j : \bigoplus_j \mathcal{N}_j \rightarrow \mathcal{M}_j
\end{equation}

\begin{equation}
|| X || = \max_{j} ||X_j||
\end{equation}
\end{lemma}

\begin{proof}
This follows from the fact that the operator norm measures how much
an operator scales any non-zero vector. If the vector comes from the
space which is the direct sum of the $\mathcal{N}_j$'s, it is in a
particular fixed subspace $\mathcal{N}_j$. Therefore, it cannot be scaled
more than the maximum operator norm of any of the $X_j$'s.
\end{proof}

First we use the result that a measuring operator $\tilde{W}$
which is the sum of
approximate measuring projectors $\P_j \otimes \tilde{U}_j$
also approximates the operator $W = \sum_{j} \Pi_{\mathcal{L}_j} \otimes U_j$
with the same error. More rigorously

%%%%%%%%%%%%%%%%%%%%%%%%%%%%%%%%%%%%%%%%%%%%%%%%%%%%%%%%%%%%%%%%%%%%%%%%%%%%%%
\begin{lemma}[Problem 12.1 \cite{ksv02}]
\label{lemma:error-sum}
Let $W$ be a unitary operator which acts on a space with two subsystems
$\mathcal{N}$ and $\mathcal{K}$, and $\tilde{W}$ which acts on the same space
but augmented with a third subsystem $\mathcal{B}^N$.

\begin{eqnarray}
W : \mathcal{N} \otimes \mathcal{K}\\
\tilde{W} : \mathcal{N} \otimes \mathcal{K} \otimes \mathcal{B}^N
\end{eqnarray}

Further, let $\mathcal{N} = \bigotimes_{j} \mathcal{L}_j$
be some orthogonal decomposition, and $U_j : \mathcal{K}$ and 
$\tilde{U}_j : \mathcal{K} \otimes \mathcal{B}^N$ be some unitary operators.
Suppose that for each $j$,
$\tilde{U}_j$ approximates (with ancillae) $U_j$ with error $\nu$
according to the
definition below.

\begin{equation}
\forall_j || \tilde{U}_j - (U_j \otimes I_{\mathcal{B}^{\otimes N}}) || \le \nu
\end{equation}

Then the measuring operator
$\tilde{W} = \sum_j \Pi_{\mathcal{L}_j} \otimes \tilde{U}_j$ approximates
with ancillae the measuring operator
$W = \sum_j \Pi_{\mathcal{L}_j} \otimes U_j$ with the same error $\nu$.

\begin{equation}
|| \tilde{W} - (W \otimes I_{\mathcal{B}^{\otimes N}}) || \le \nu ||
\end{equation}

\end{lemma}

\begin{proof}
We make use of the standard embedding
$V:\mathcal{K} \rightarrow \mathcal{K} \otimes \mathcal{B}^N$ which simply
augments a state in $\mathcal{K}$ with ancillae in the state $\ket{0}^N$.
That is, $V\ket{\zeta} = \ket{\zeta} \otimes \ket{0}^N$.

\begin{equation}
\forall_j || \tilde{U}_j V - V U_j || \le \delta
\end{equation}

Then $\tilde{W}$ \emph{with} ancillae approximates $\tilde{W}$
\emph{without} ancillae
according to the following equation.

\begin{equation}
|| \tilde{W} (I_{\mathcal{N}} \otimes V) - (I_{\mathcal{N}} \otimes V) W || \le \nu
\end{equation}

\end{proof}

We now return to the setting of Section \ref{subsec:approx} and repeat
its definitions here. Given a measuring operator $W$ which approximately
measures a function, what is the error of approximation
of a two stage measurement using $W$ and intermediate ancillae?
We will see later that such an approximate two-stage measurement
corresponds to parallelized phase estimation.

%%%%%%%%%%%%%%%%%%%%%%%%%%%%%%%%%%%%%%%%%%%%%%%%%%%%%%%%%%%%%%%%%%%%%%%%%%%%%%
\begin{theorem}[Problem 12.2, \cite{ksv02}]
We consider the space
$\mathcal{N} \otimes \mathcal{B}^{\otimes N} \otimes \mathcal{K}$ with the
following orthogonal subsystem decompositions:
$\mathcal{N} = \bigoplus_{j \in \Omega} \mathcal{L}_j$ and
$\mathcal{B}^{\otimes N} = \bigoplus_{y \in \Delta} \mathcal{M}_y$. We
define two operators, $W$ which measures $\mathcal{N}$ as object into
$\mathcal{B}^{\otimes N}$ as instrument and $V$ which measures
$\mathcal{B}^{\otimes N}$ as object into $\mathcal{K}$ as instrument.

\begin{eqnarray}
W : \mathcal{N} \otimes \mathcal{K}\\
\tilde{W} : \mathcal{N} \otimes \mathcal{K} \otimes \mathcal{B}^N
\end{eqnarray}

If $W$ approximately measures a function $f : \Omega \rightarrow \Delta$
with error $\epsilon$, then the operator $\tilde{Y} = W^{-1}VW$ approximates
the following operator $Y$ with error $2\sqrt{\epsilon}$.

\end{theorem}

\begin{proof}

Alternatively, we can take the norm of the vector difference $\ket{\psi}$
between an
arbitrary state $\ket{\xi} \mathcal{N}$ operated on by $W$ and the
state $\ket{\xi} \otimes \ket{0^N}$ operated on by $\tilde{W}$.

\begin{equation}
\ket{\eta} = W\ket{\xi} \qquad
\ket{\tilde{\eta}} = \tilde{W}(\ket{\xi} \otimes \ket{0^N}) \qquad
\ket{psi} = \ket{\tilde{\eta}} - \ket{\eta} \otimes \ket{0^N}
\end{equation}

When we take the vector norm of

\end{proof}

%%%%%%%%%%%%%%%%%%%%%%%%%%%%%%%%%%%%%%%%%%%%%%%%%%%%%%%%%%%%%%%%%%%%%%%%%%%%%%
\subsection{Error Probability With Projective Measurement}
\label{subsec:error-proj}
