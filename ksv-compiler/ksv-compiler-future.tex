\section{Conclusion and Future Directions}

In summary, we have seen the expected asymptotic bounds of both the
Solovay-Kitaev and KSV quantum compiling algorithms reflected in
numerical resource comparisons. Solovay-Kitaev requires a large classical
preprocessing overhead but produces a more tractable number of compiled
gates and zero ancillae, albeit at larger circuit depth. This seems to be
a more reasonable choice for early experiments in running algorithms on an
80-qubit ion-trap quantum computer, where physical trapping constraints
make large numbers of qubits problematic but we are potentially willing to
wait a long time for the computation to complete. On the other hand, the
situation may change in the future when
quantum computers become more mature, scalable, and parallel;
when we become more ambitious in our
algorithm input sizes; and when the performance bottleneck becomes circuit depth.
In that case, KSV may be preferrable.
Quantum computer engineers of the future will be able to use this work to
choose the most suitable quantum compiling technique currently available
as well as compare it to future compiling algorithms.

Moreover, the development of a compiler is intertwined with the
development of the underlying architecture. Just as classical programming
languages hint at what features would be desirable to move out of software
(and compile-time)
and into hardware (and run-time), Super-Kitaev provides some suggestions
to future quantum architectures. Hardware which seeks to take advantage of
this low-depth compiling would need to have a
"phase factory" for enacting the $\Lambda(e^{i\phi})$ gate, which would
included an efficient parallelized phase estimation routine. These are
novel resources which are currently not being considered in related literature.
