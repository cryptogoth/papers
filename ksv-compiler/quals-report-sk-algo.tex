\section{Review of Solovay-Kitaev}
\label{sec:sk-algo}

The essential structure of SK is to recursively generate nets of
successively finer precision and to use a divide-and-conquer approach to
approximate the desired gate. At any given level of recursion, the input 
gate is divided up into two halves which are closer to identity but can be
approximated with less precision. Our pseudo-code and explanation
follows the excellent exposition in \cite{Dawson2005} and \cite{harrow01}

\begin{algorithmic}[1]
\STATE \textsc{function} $\tilde{U}_i \leftarrow$ SK$(U,n)$
\IF{$i == 0$}
\STATE $\tilde{U}_i \leftarrow $ BASIC-APPROX$(U)$
\ELSE
\STATE $\tilde{U}_{i-1} \leftarrow$ SK$(U, i-1)$
\STATE $A,B \leftarrow $ FACTOR$(U\tilde{U}^\dagger_{i-1})$
\STATE $\tilde{A}_{i-1} \leftarrow $ SK$(A, i-1)$
\STATE $\tilde{B}_{i-1} \leftarrow $ SK$(B, i-1)$
\STATE $\tilde{U}_i \leftarrow \tilde{A}_{i-1}\tilde{B}_{i-1}\tilde{A}^\dagger_{i-1}\tilde{B}^\dagger_{i-1}\tilde{U}_{i-1}$
\ENDIF
\STATE return $\tilde{U}_i$
\end{algorithmic}

Since the recursion must eventually bottom out, we must precompute some sequences
of gates from $\mathcal{G}$ up to length $l_0$. This is the classical
preprocessing step for SK which requires upfront storage space for this
coarsest-grained net, where
each sequence is no more than $\epsilon_0$ from its nearest neighbor. According
to \cite{Dawson2005} the values of $l_0=16$ and $\epsilon_0 = 0.14$ using
operator norm distance is sufficient for most applications.
This step can be
done offline and reused across multiple runs of the compiler, assuming
$\mathcal{G}$ for your quantum computer doesn't change.

The BASIC-APPROX function above does a lookup (e.g. using some kd-tree search
maneuvers through higher-dimensional vector spaces) using this $\epsilon_0$-net,
and all higher recursive calls to SK are effectively constructing
finer $\epsilon$-nets on the fly as needed.

The FACTOR function performs a balanced group commutator decomposition,
$U = ABA^\dagger B^\dagger$, and then recursively approximates the $A$ and $B$
operators, again using SK. We denote by $\tilde{U}_i$ as the approximation
of $U$ using $i$ levels of SK recursion. Intuitively, when they are multiplied
together again, along with their inverses, their errors (which go like
$\epsilon$) are symmetric and cancel out in such a way that the resulting
product $U$ has errors which go like $\epsilon^2$. In this manner, we can
eventually sharpen our desired error down to any value. We use the
geometric decomposition in \cite{Dawson2005} rather than the decomposition
based on a BCH-like approximation as in \cite{harrow01}, although it is not
known which method converges more quickly to a desired gate in general.

Each level $i$ of recursion solves the problem of compiling 
$U_i$ by combining five gates compiled at the lower $i-1$ level.
Therefore, the compiled circuit size is upper-bounded by $5^i$, and
this is in general the same as the circuit depth (since not all gates in
$\mathcal{G}$ commute).
