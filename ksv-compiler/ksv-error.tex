\documentclass{article}
\usepackage{eprint}
\usepackage{amsfonts}
\usepackage[osf]{mathpazo}
\usepackage{amsthm}

%    Q-circuit version 1.06
%    Copyright (C) 2004  Steve Flammia & Bryan Eastin

%    This program is free software; you can redistribute it and/or modify
%    it under the terms of the GNU General Public License as published by
%    the Free Software Foundation; either version 2 of the License, or
%    (at your option) any later version.
%
%    This program is distributed in the hope that it will be useful,
%    but WITHOUT ANY WARRANTY; without even the implied warranty of
%    MERCHANTABILITY or FITNESS FOR A PARTICULAR PURPOSE.  See the
%    GNU General Public License for more details.
%
%    You should have received a copy of the GNU General Public License
%    along with this program; if not, write to the Free Software
%    Foundation, Inc., 59 Temple Place, Suite 330, Boston, MA  02111-1307  USA

\usepackage[matrix,frame,arrow]{xy}
\usepackage{amsmath}
\newcommand{\bra}[1]{\left\langle{#1}\right\vert}
\newcommand{\ket}[1]{\left\vert{#1}\right\rangle}
    % Defines Dirac notation.
\newcommand{\qw}[1][-1]{\ar @{-} [0,#1]}
    % Defines a wire that connects horizontally.  By default it connects to the object on the left of the current object.
    % WARNING: Wire commands must appear after the gate in any given entry.
\newcommand{\qwx}[1][-1]{\ar @{-} [#1,0]}
    % Defines a wire that connects vertically.  By default it connects to the object above the current object.
    % WARNING: Wire commands must appear after the gate in any given entry.
\newcommand{\cw}[1][-1]{\ar @{=} [0,#1]}
    % Defines a classical wire that connects horizontally.  By default it connects to the object on the left of the current object.
    % WARNING: Wire commands must appear after the gate in any given entry.
\newcommand{\cwx}[1][-1]{\ar @{=} [#1,0]}
    % Defines a classical wire that connects vertically.  By default it connects to the object above the current object.
    % WARNING: Wire commands must appear after the gate in any given entry.
\newcommand{\gate}[1]{*{\xy *+<.6em>{#1};p\save+LU;+RU **\dir{-}\restore\save+RU;+RD **\dir{-}\restore\save+RD;+LD **\dir{-}\restore\POS+LD;+LU **\dir{-}\endxy} \qw}
    % Boxes the argument, making a gate.
\newcommand{\meter}{\gate{\xy *!<0em,1.1em>h\cir<1.1em>{ur_dr},!U-<0em,.4em>;p+<.5em,.9em> **h\dir{-} \POS <-.6em,.4em> *{},<.6em,-.4em> *{} \endxy}}
    % Inserts a measurement meter.
\newcommand{\measure}[1]{*+[F-:<.9em>]{#1} \qw}
    % Inserts a measurement bubble with user defined text.
\newcommand{\measuretab}[1]{*{\xy *+<.6em>{#1};p\save+LU;+RU **\dir{-}\restore\save+RU;+RD **\dir{-}\restore\save+RD;+LD **\dir{-}\restore\save+LD;+LC-<.5em,0em> **\dir{-} \restore\POS+LU;+LC-<.5em,0em> **\dir{-} \endxy} \qw}
    % Inserts a measurement tab with user defined text.
\newcommand{\measureD}[1]{*{\xy*+=+<.5em>{\vphantom{#1}}*\cir{r_l};p\save*!R{#1} \restore\save+UC;+UC-<.5em,0em>*!R{\hphantom{#1}}+L **\dir{-} \restore\save+DC;+DC-<.5em,0em>*!R{\hphantom{#1}}+L **\dir{-} \restore\POS+UC-<.5em,0em>*!R{\hphantom{#1}}+L;+DC-<.5em,0em>*!R{\hphantom{#1}}+L **\dir{-} \endxy} \qw}
    % Inserts a D-shaped measurement gate with user defined text.
\newcommand{\multimeasure}[2]{*+<1em,.9em>{\hphantom{#2}} \qw \POS[0,0].[#1,0];p !C *{#2},p \drop\frm<.9em>{-}}
    % Draws a multiple qubit measurement bubble starting at the current position and spanning #1 additional gates below.
    % #2 gives the label for the gate.
    % You must use an argument of the same width as #2 in \ghost for the wires to connect properly on the lower lines.
\newcommand{\multimeasureD}[2]{*+<1em,.9em>{\hphantom{#2}}\save[0,0].[#1,0];p\save !C *{#2},p+LU+<0em,0em>;+RU+<-.8em,0em> **\dir{-}\restore\save +LD;+LU **\dir{-}\restore\save +LD;+RD-<.8em,0em> **\dir{-} \restore\save +RD+<0em,.8em>;+RU-<0em,.8em> **\dir{-} \restore \POS !UR*!UR{\cir<.9em>{r_d}};!DR*!DR{\cir<.9em>{d_l}}\restore \qw}
    % Draws a multiple qubit D-shaped measurement gate starting at the current position and spanning #1 additional gates below.
    % #2 gives the label for the gate.
    % You must use an argument of the same width as #2 in \ghost for the wires to connect properly on the lower lines.
\newcommand{\control}{*-=-{\bullet}}
    % Inserts an unconnected control.
\newcommand{\controlo}{*!<0em,.04em>-<.07em,.11em>{\xy *=<.45em>[o][F]{}\endxy}}
    % Inserts a unconnected control-on-0.
\newcommand{\ctrl}[1]{\control \qwx[#1] \qw}
    % Inserts a control and connects it to the object #1 wires below.
\newcommand{\ctrlo}[1]{\controlo \qwx[#1] \qw}
    % Inserts a control-on-0 and connects it to the object #1 wires below.
\newcommand{\targ}{*{\xy{<0em,0em>*{} \ar @{ - } +<.4em,0em> \ar @{ - } -<.4em,0em> \ar @{ - } +<0em,.4em> \ar @{ - } -<0em,.4em>},*+<.8em>\frm{o}\endxy} \qw}
    % Inserts a CNOT target.
\newcommand{\qswap}{*=<0em>{\times} \qw}
    % Inserts half a swap gate. 
    % Must be connected to the other swap with \qwx.
\newcommand{\multigate}[2]{*+<1em,.9em>{\hphantom{#2}} \qw \POS[0,0].[#1,0];p !C *{#2},p \save+LU;+RU **\dir{-}\restore\save+RU;+RD **\dir{-}\restore\save+RD;+LD **\dir{-}\restore\save+LD;+LU **\dir{-}\restore}
    % Draws a multiple qubit gate starting at the current position and spanning #1 additional gates below.
    % #2 gives the label for the gate.
    % You must use an argument of the same width as #2 in \ghost for the wires to connect properly on the lower lines.
\newcommand{\ghost}[1]{*+<1em,.9em>{\hphantom{#1}} \qw}
    % Leaves space for \multigate on wires other than the one on which \multigate appears.  Without this command wires will cross your gate.
    % #1 should match the second argument in the corresponding \multigate. 
\newcommand{\push}[1]{*{#1}}
    % Inserts #1, overriding the default that causes entries to have zero size.  This command takes the place of a gate.
    % Like a gate, it must precede any wire commands.
    % \push is useful for forcing columns apart.
    % NOTE: It might be useful to know that a gate is about 1.3 times the height of its contents.  I.e. \gate{M} is 1.3em tall.
    % WARNING: \push must appear before any wire commands and may not appear in an entry with a gate or label.
\newcommand{\gategroup}[6]{\POS"#1,#2"."#3,#2"."#1,#4"."#3,#4"!C*+<#5>\frm{#6}}
    % Constructs a box or bracket enclosing the square block spanning rows #1-#3 and columns=#2-#4.
    % The block is given a margin #5/2, so #5 should be a valid length.
    % #6 can take the following arguments -- or . or _\} or ^\} or \{ or \} or _) or ^) or ( or ) where the first two options yield dashed and
    % dotted boxes respectively, and the last eight options yield bottom, top, left, and right braces of the curly or normal variety.
    % \gategroup can appear at the end of any gate entry, but it's good form to pick one of the corner gates.
    % BUG: \gategroup uses the four corner gates to determine the size of the bounding box.  Other gates may stick out of that box.  See \prop. 
\newcommand{\rstick}[1]{*!L!<-.5em,0em>=<0em>{#1}}
    % Centers the left side of #1 in the cell.  Intended for lining up wire labels.  Note that non-gates have default size zero.
\newcommand{\lstick}[1]{*!R!<.5em,0em>=<0em>{#1}}
    % Centers the right side of #1 in the cell.  Intended for lining up wire labels.  Note that non-gates have default size zero.
\newcommand{\ustick}[1]{*!D!<0em,-.5em>=<0em>{#1}}
    % Centers the bottom of #1 in the cell.  Intended for lining up wire labels.  Note that non-gates have default size zero.
\newcommand{\dstick}[1]{*!U!<0em,.5em>=<0em>{#1}}
    % Centers the top of #1 in the cell.  Intended for lining up wire labels.  Note that non-gates have default size zero.
\newcommand{\Qcircuit}{\xymatrix @*=<0em>}
    % Defines \Qcircuit as an \xymatrix with entries of default size 0em.

\theoremstyle{definition} \newtheorem{lemma}{Lemma}
\theoremstyle{definition} \newtheorem{theorem}{Theorem}

\title{Measurement Error in KSV Phase Estimation}

\author{Paul Pham}

\begin{document}

\maketitle

In this note, we provide a detailed calculation for the error in
phase estimation due to Kitaev, Shen, and Vyalyi (KSV) where projective
measurement is used instead of a coherent measurement. We show an
increase from $2\sqrt{\epsilon}$ to $\sqrt{2}\sqrt[4]{\epsilon}$. The use
of projective measurement is to offload many trigonometric operations
onto a classical controller instead of doing them reversibly on a quantum
computer. The cost of projective measurement is leaving garbage qubits in
the ancillae of the target register, but this is a constant amount relative to
the size of circuits we may wish to compile using the KSV procedure.
Our goal is to show that this increase in error by a square root
factor is negligible and may be preferrable in realistic implementations.

To begin with, we review (coherent) measurement operators as a generalization of
controlled quantum operators, and then we further extend them to the case
of approximately measuring functions on orthogonal basis decompositions
where there are garbage bits left over in an ancillary register which must
be uncomputed.
Next, we calculate the error of this approximate measurement with ancillae.
Then, we show how this general measurement procedure corresponds to
estimating the phase of the modular multiplication operator used in
Shor's factoring algorithm. Finally, we show that the error only
increases by a square root factor when we projectively measure the garbage
in the ancillae instead of coherently simulating the measurement.

%%%%%%%%%%%%%%%%%%%%%%%%%%%%%%%%%%%%%%%%%%%%%%%%%%%%%%%%%%%%%%%%%%%%%%%%%%%%%%
%%%%%%%%%%%%%%%%%%%%%%%%%%%%%%%%%%%%%%%%%%%%%%%%%%%%%%%%%%%%%%%%%%%%%%%%%%%%%%
\section{Measurement Operators}
\label{sec:meas-ops}

First we introduce some preliminary definitions related to measurement.

A very common quantum operation entangles the results of one register
(called the \emph{target}) based on the value of another register (called the
\emph{control}). The most basic case of this is CNOT, or $\Lambda(X)$,
operator, which operates
on a control register of one qubit and a target register of one qubit.

\begin{equation}
\Lambda(X)\ket{y,z} \rightarrow \ket{y, z \oplus y}
\end{equation}

We have two ways of describing how CNOT is measuring in this case.
We can say CNOT is \emph{measuring with respect to the decomposition} of
the control registers, namely the computational basis, in that the
operation on the target register (flipping the bit) depend on decomposing the
control register in the basis $\{0,1\}$. We can also say that CNOT is
\emph{measuring a function} $f:\{0,1\} \rightarrow \{0,1\}$ which in this
case is simply the copy operator, $f(x) = x$. We can rewrite the equation
above as:

\begin{equation}
\Lambda(X)\ket{y,z} \rightarrow \ket{y, z \oplus f(y)}
\end{equation}

However, we know that the distinction between control and target registers
often depends on a particular basis. For example, we can flip the direction
of control and target in the CNOT case by conjugating both qubits with
Hadamard operators. The general characteristic of measurement operators
is that they are entangling, and that they can encode information about
one register in another in a very general way.
From the point of view of measurement, we call the control register the
\emph{object}, as in the state we are trying to measure, and the target register
is the \emph{instrument}, as in the state that we transform according to
projecting the measurement object in some orthogonal decomposition.

Let's begin with a simple but more general case of a measuring operator
$W$ which operates on a space decomposed into the subsystems $\mathcal{N}$
(the measurement object) and $\mathcal{K}$ (the measurement instrument)
according to the orthogonal decomposition
$\mathcal{N} = \bigoplus_j \mathcal{L}_j$, so-called because
each of the $\mathcal{L}_j$ are pairwise orthogonal subspaces.
$W$ will perform a different unitary
operator $U_j$ on the subsystem $\mathcal{K}$
depending on the projection of $\mathcal{N}$ into each $\mathcal{L}_j$.

\begin{equation}
W = \sum_{j \in \Omega} \Pi_{\mathcal{L}_j} \otimes U_j
\end{equation}

An interesting fact about this definition of measurement operators is
that approximativeness is preserved. If each unitary $U_j$ is replaced with
another unitary $\tilde{U}_j$ that approximates it with precision $\delta$,
then the new measuring operator $\tilde{W}$ also approximates the original
$W$ with the precision $\delta$.

This precision is defined in terms of the inner product between any state
$\ket{\zeta}$
operated
on by $W$ and by $\tilde{W}$. We can decompose $\ket{\zeta}$ into
two subsystems $\ket{\psi}$ and $\ket{\phi}$
corresponding to the spaces $\mathcal{N}$ and $\mathcal{K}$
above. We will use this fact later.

\begin{equation}
\bra{\zeta} W^{\dag} \tilde{W} \ket{\zeta} =
\bra{\psi} \otimes \bra{\phi} W^{\dag} \tilde{W} \ket{\psi} \otimes \ket{\phi} =
\sum_{j \in \Omega} \left( \bra{\psi} \Pi_{\mathcal{L}_j} \ket{\psi} \otimes
\bra{\phi} U^{\dag}_j \tilde{U}_j \ket{\phi} \right) \le \delta
\end{equation}

%%%%%%%%%%%%%%%%%%%%%%%%%%%%%%%%%%%%%%%%%%%%%%%%%%%%%%%%%%%%%%%%%%%%%%%%%%%%%%
%%%%%%%%%%%%%%%%%%%%%%%%%%%%%%%%%%%%%%%%%%%%%%%%%%%%%%%%%%%%%%%%%%%%%%%%%%%%%%
\section{Operators That Measure a Function}
\label{sec:meas-func}

We now introduce the idea that an operator can measure a function
from the indices of the measurement object space $\mathcal{N}$
to the measurement instrument space $\mathcal{K}$ with respect to
some fixed orthogonal decompositions of these spaces.

\begin{equation}
\mathcal{N} = \bigotimes_{j \in \Omega} \qquad
\mathcal{K} = \bigotimes_{y \in \Delta} \qquad
f:\Omega \rightarrow \Delta
\end{equation}

Saying that an operator $Y$ is measuring with respect to a fixed
orthogonal decomposition $\Omega$ is equivalent to saying that
$Y$ measures the function $f$, which need not even be reversible.
The simplest, but not the most general, form of such an operator projects
the first subsystem $\mathcal{N}$ into a subspace $\mathcal{L}_j$
and performs the
corresponding operation $Q_{f(j)}$ on the second subsystem $\mathcal{K}$.
In Equation \ref{eqn:y-ideal}, we define $Y$ as the sum of these projectors,
and our notation means it operates on the space of $\mathcal{N}$
tensored with $\mathcal{K}$.

\begin{equation}
Y = \sum_{j \in \Omega} \Pi_{\mathcal{L}_j} \otimes Q_{f(j)} : \mathcal{N} \otimes \mathcal{K}
\label{eqn:y-ideal}
\end{equation}

%%%%%%%%%%%%%%%%%%%%%%%%%%%%%%%%%%%%%%%%%%%%%%%%%%%%%%%%%%%%%%%%%%%%%%%%%%%%%%
%%%%%%%%%%%%%%%%%%%%%%%%%%%%%%%%%%%%%%%%%%%%%%%%%%%%%%%%%%%%%%%%%%%%%%%%%%%%%%
\section{Measurement Operators with Ancillae}

However, this is not the most general case of a measurement operator.
We make \emph{three extensions} here which require cascading
two rounds of measurement through
an ancillary register. We now define a new
composite measurement operator $\tilde{Y}$
on this space with three subsystems corresponding to the input register
$\mathcal{N}$,
the intermediate ancillary register $\mathcal{B}^N$, and a final output register
$\mathcal{K}$ which approximates $Y$ from Equation \ref{eqn:y-ideal}.

\begin{equation}
\tilde{Y} : \mathcal{N} \otimes \mathcal{B}^N \otimes \mathcal{K}
\end{equation}

We will build up to an operational definition of $\tilde{Y}$ later, but first
we must motivate what it must achieve.

In the first round, we measure with our object in $\mathcal{N}$ and our
instrument in $\mathcal{B}^N$. Here we allow for garbage, which is our first
extension, described in \ref{subsec:garbage}.
In the second round, we measure with our object being the
instrument of the first round, in $\mathcal{B}^N$, and our instrument is
in $\mathcal{K}$. Here we allow for operations other than just copying from
$\mathcal{B}^N$ to $\mathcal{K}$, which is our second extension, described
in \ref{subsec:non-copy}. Finally, instead of implementing our ideal
$Y$ directly, we allow ourselves, and quantify what it means, to
approximate it as $\tilde{Y}$.

%%%%%%%%%%%%%%%%%%%%%%%%%%%%%%%%%%%%%%%%%%%%%%%%%%%%%%%%%%%%%%%%%%%%%%%%%%%%%%
\subsection{First Extension: Measurement with Garbage}
\label{subsec:garbage}

To motivate why we need this new composite measurement $Y$ instead of just
using $W$ directly from the previous section, let's consider the problem of
garbage.
In general, only some of the information computed by $W$ into its
target register $\mathcal{K}$ above may be useful, but we generate some
garbage qubits (say, to make the operation reversible).
To be concrete, let's say that we have the following:

\begin{equation}
W: \mathcal{N} \otimes \mathcal{B}^N =
\sum_{j \in \Omega} \Pi_{\mathcal{L}_j} \otimes U_j \qquad
U_j : \mathcal{B}^N \rightarrow \mathcal{B}^N \qquad
U_j\ket{0} = \sum_{y,z} c_{y,z}(j)\ket{y,z}
\end{equation}

where $y \in \mathbb{B}^m$ represents the useful part of the result,
$z \in \mathbb{B}^{N-m}$ is garbage, and the complex weights
$c_{y,z} \in \mathbb{C}$
are functions of the index of the orthogonal decomposition of $\mathcal{N}$,
the number $j$. In many cases, we would like to uncompute this garbage in
order to use space efficiently.
We can do this using an uncomputing trick due to Charlie Bennett
\cite{Bennett1973} by first running an operation $U$ which may produce
garbage, copying out the useful part of the result to a
second register, and finally running $U^{\dagger}$ to uncompute the
input register.

Now we will give a more concrete definition for $Y$ in terms of
$W$ and the operation $V$ which simply copies a state $\ket{\psi} \in \mathcal{B}^N$
from the ancillary register to a state $\ket{\phi} in \mathcal{K}$ in the
output register using bitwise CNOTs, or bitwise addition modulo $2^n$ for an
$n$-qubit register. Then $Y = W^{-1}VW$, where $W$ is our operator from above
which computes a function with garbage, $V$ is our copy operation, and
$W^{-1}$ uncomputes the original function and it garbage. We make this
more rigorous in Equation \ref{eqn:wvw}.

\begin{eqnarray}
Y & = & WVW^{-1} : \mathcal{N} \otimes \mathcal{B}^N \otimes \mathcal{K} \\
W & = &
\sum_{j \in \Omega} \Pi_{\mathcal{L}_j} \otimes U_j \otimes I_{mathcal{K}}\\
V & : & \ket{x} \otimes \ket{y} \otimes \ket{z} \rightarrow \ket{x} \otimes \ket{y} \otimes \ket{z \oplus y}
\label{eqn:wvw}
\end{eqnarray}

%%%%%%%%%%%%%%%%%%%%%%%%%%%%%%%%%%%%%%%%%%%%%%%%%%%%%%%%%%%%%%%%%%%%%%%%%%%%%%
\subsection{Second Extension: Measurement with a Non-Copy Operation}
\label{subsec:non-copy}

Our notion of measurement with garbage is still not the most general it could be.
For example, there is no reason why the decomposition of $\mathcal{K}$ has to
be the same as that for $\mathcal{B}^N$ or $\mathcal{N}$, or why we are limited
to strictly copying the useful output from $\mathcal{B}^N$ into $\mathcal{K}$.
We don't need to limit the function $f$ measured by $W$ to be invertible.
Furthermore, for practical reasons, there may be a more efficient way of encoding
the useful part of $f$'s output, or we may need to do further processing on it
as part of the algorithm we want to run.

To allow this, in this section, we allow an arbitrary orthogonal decomposition
$\mathcal{K} = \bigotimes_{y \in \Delta} \mathcal{M}_y$. This is the same as
allowing $V$ to be an arbitrary sum of projectors onto $\{\mathcal{M}_y\}$
and corresponding unitary operators $Q_y$ on $\mathcal{K}$.

\begin{equation}
V = \sum_{y \in \Delta} I_{\mathcal{N}} \otimes Q_y \otimes \Pi_{\mathcal{M}_y}
\end{equation}

%%%%%%%%%%%%%%%%%%%%%%%%%%%%%%%%%%%%%%%%%%%%%%%%%%%%%%%%%%%%%%%%%%%%%%%%%%%%%%
\subsection{Third Extension: Approximate Measurement}
\label{subsec:approx}

We can now allow a very general form of
measurement which allows garbage, non-copying uncomputation, and finally,
approximating a function. We will explain this last feature in this section.

We review that in the last two subsections we have been building up a
composite measurement operator $\tilde{Y} = W^{-1}VW$ which consists of 
these two operations on a space with three subsystems.

\begin{equation}
W = \sum_{j \in \Omega} \Pi_{\mathcal{L}_j} \otimes R_j \otimes I_{\mathcal{K}} \qquad
V = \sum_{y \in \Delta} I_{\mathcal{N}} \otimes \Pi_{\mathcal{M}_y} \otimes Q_y
\end{equation}

We now reveal that $\tilde{Y}$ approximates $Y$ from Equation \ref{eqn:y-ideal}.
What does this mean? For one operator, say $\tilde{X}$, to approximate another,
say $X$, within an error $\delta$ means that the minimum overlap between
any state $\ket{\eta}$ that is operated on by $\tilde{X}$ and $X$ is at least
$1 - \delta$. This is shown in Equation \ref{eqn:overlap}.

\begin{equation}
\bra{\eta} \tilde{X}^{\dagger} X \ket{\eta} \ge 1 - \delta
\end{equation}

Now how do we relate this to approximating a function
$f$ between the orthogonal decomposition indices of $\mathcal{N}$ and 
$\mathcal{K}$? And what does this latter concept mean?
It means that the conditional probabilities of
getting a given output index $y \in \Delta$ given an input index $j \in \Omega$,
namely $P(y \mid j) = \bra{0^N} R^{\dagger}_j \Pi_{\mathcal{M}_y} R_j \ket{0^N}$
satisfies Equation \ref{eqn:f-approx}. Approximating a function $f$ with
error probability $\epsilon$ means that the probability of getting the
correct answer $f(j)$ is bounded below by $1 - \epsilon$, and the probability
of getting any $y \ne f(j)$ is bounded above by $\epsilon$.

\begin{eqnarray}
P(f(j) \mid j) & \ge & 1 - \epsilon \\
\sum_{y \ne f(j)} P(y \mid j) & < & \epsilon
\label{eqn:f-approx}
\end{eqnarray}

In the next section, we will delve into detailed calculations of relating
$\delta$ and $\epsilon$ given the framework we have built up until now.

%%%%%%%%%%%%%%%%%%%%%%%%%%%%%%%%%%%%%%%%%%%%%%%%%%%%%%%%%%%%%%%%%%%%%%%%%%%%%%
%%%%%%%%%%%%%%%%%%%%%%%%%%%%%%%%%%%%%%%%%%%%%%%%%%%%%%%%%%%%%%%%%%%%%%%%%%%%%%
\section{Error Calculations}

In this section, we will calculate two error probabilities $\delta$ for
an approximate measuring operator $\tilde{Y}$, given that we are able to
approximate a function $f$ with error probability $\epsilon$. The first
error probability, which follows exactly the development in Problem 12.2
of Ref \cite{Kitaev2002}, assumes that we measure \emph{coherently}, meaning
that at no point do we projectively measure, and we are able to perfectly
uncompute all garbage. This is the ideal case, which we calculate in
\ref{subsec:error-noproj}. The second error probability, which we use in
\cite{Pham2012a}, involves projectively measuring as part of the operator $W$,
which involves some purely classical reversible operations that can be
offloaded to a classical controller. Afterwards, we execute $V$ as before,
but it is now impossible to run $W^{-1}$ because of the projective measurement.
This leaves some amount of garbage in the ancillary
subsystem $\mathcal{B}^N$, and we calculate it in \ref{subsec:error-proj}.

%%%%%%%%%%%%%%%%%%%%%%%%%%%%%%%%%%%%%%%%%%%%%%%%%%%%%%%%%%%%%%%%%%%%%%%%%%%%%%
\subsection{Error Probability Without Projective Measurement}
\label{subsec:error-noproj}

First, we use this preliminary lemma without proof, which is a property of
the operator norm.

\begin{lemma}
\label{lemma:sum-norm}
Let $X_j \in \$ be a collection of unitary operators,
\end{lemma}

First we use the result that a measuring operator $\tilde{W}$
which is the sum of
approximate measuring projectors $\Pi_j \otimes \tilde{U}_j$
also approximates the operator $W = \sum_{j} \Pi_{\mathcal{L}_j} \otimes U_j$
with the same error. More rigorously

\begin{lemma}
\label{lemma:error-sum}
Let $W$ be a unitary operator which acts on a space with two subsystems
$\mathcal{N}$ and $\mathcal{K}$, and $\tilde{W}$ which acts on the same space
but augmented with a third subsystem $\mathcal{B}^N$.

\begin{eqnarray}
W : \mathcal{N} \otimes \mathcal{K}\\
\tilde{W} : \mathcal{N} \otimes \mathcal{K} \otimes \mathcal{B}^N
\end{eqnarray}

Further, let $\mathcalN} = \bigotimes_{j} \mathcal{L}_j$
be some orthogonal decomposition, and $U_j : \mathcal{K}$ and 
$\tilde{U}_j : \mathcal{K} \otimes \mathcal{B}^N$ be some unitary operators
which act on their respective spaces. Suppose that for each $j$,
$\tilde{U}_j$ approximates $U_j$ with precision $\nu$ according to the
definition below. We make use of the standard embedding
$V:\mathcal{K} \rightarrow \mathcal{K} \otimes \mathcal{B}^N$ which simply
augments a state in $\mathcal{K}$ with ancillae in the state $\ket{0}^N$.
That is, $V\ket{\zeta} = \ket{\zeta} \otimes \ket{0}^N$.

\begin{equation}
\forall_j || \tilde{U}_j V - V U_j || \le \delta
\end{equation}

Then $\tilde{W}$ with ancillae approximates $\tilde{W}$ without ancillae
according to the following equation.

\begin{equation}
|| \tilde{W} (I_{\mathcal{N} \otimes V} - (I_{\mathcal{N}} \otimes V) W || \le \nu
\end{equation}

\end{lemma}

\begin{proof}
We first use the following fact about 
\end{proof}

%%%%%%%%%%%%%%%%%%%%%%%%%%%%%%%%%%%%%%%%%%%%%%%%%%%%%%%%%%%%%%%%%%%%%%%%%%%%%%
\subsection{Error Probability With Projective Measurement}
\label{subsec:error-proj}

\bibliography{ksv-error}
\bibliographystyle{tocplain}

\end{document}