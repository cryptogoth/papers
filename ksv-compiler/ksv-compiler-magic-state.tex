\subsection{Addition Eigenstates}
\label{subsec:magic-state}

Suppose you were given the state $\ket{\psi_{n,k}}$
parameterized by the
integers $n$ and $k$ such that

\begin{equation}
\ket{\psi_{n,k}} = \frac{1}{\sqrt{2^n}} \sum_{j=0}^{2^n-1}
e^{-2\pi i j k / 2^n} \ket{j}
\end{equation}

where $0 < k < 2^n$ and $k$ is odd.

Note that these states $\ket{\psi}$ are the
QFT states of the computational basis.
%%%%%%%%%%%%%%%%%%%%%%%%%%%%%%%%%%%%%%%%%%%%%%%%%%%%%%%%%%%%%%%%%%%%%%%%%%%%%%%
%%%%%%%%%%%%%%%%%%%%%%%%%%%%%%%%%%%%%%%%%%%%%%%%%%%%%%%%%%%%%%%%%%%%%%%%%%%%%%%
% actually there is a more precise description here. QFT over 2^n ?
This implicitly argues that these states are hard to create,
because QFT in general requires
$O(n^2)$-depth and we need a logarithmic-depth compiler.
%%%%%%%%%%%%%%%%%%%%%%%%%%%%%%%%%%%%%%%%%%%%%%%%%%%%%%%%%%%%%%%%%%%%%%%%%%%%%%%
%%%%%%%%%%%%%%%%%%%%%%%%%%%%%%%%%%%%%%%%%%%%%%%%%%%%%%%%%%%%%%%%%%%%%%%%%%%%%%%
% we should mention the difficulty even of approximate QFT
% and use circuit size, not depth since constant-depth QFT was shown in
% Brown, Kasheffi, Perdrix (cite here too, just to show off)

However, what we can do is create a superposition over all odd $k=(2s-1)$
by starting in the state $\ket{0}^{\otimes n}$,
then applying a Hadamard and $\sigma^z$ to the most significant qubit.

\begin{equation}
\ket{\eta} = \normtwo \ket{0} - \normtwo \ket{2^{n-1}} =
\frac{1}{\sqrt{2^{n-1}}} \sum_{s=1}^{2^{n-1}} \ket{\psi_{n,2s-1}}
\end{equation}

More obviously relevant to our overall goal of approximating
$\Lambda(e^{i\phi})$, we can enact a phase
shift simply by performing the following modular addition operator, for
which $\ket{\psi_{n,k}}$ are eigenstates.

\begin{equation}
A\ket{j} \rightarrow \ket{j+1 \bmod 2^n}
\end{equation}

Applying this operator to its eigenstates results in a phase shift which
depends on the particular eigenstate.
 
\begin{equation}
A\ket{\psi_{n,k}} = e^{2\pi i \phi_k} \ket{\psi_{n,k}}
\end{equation}

Finding the eigenvalue $e^{2\pi i \phi_k}$ corresponds to finding
the phase $\phi_k = k / 2^n$.
Repeated application of $A$ (say $p$ times) would result in a phase
added to the eigenstate equal to a multiple of $e^{2\pi i p / 2^n}$

\begin{equation}
A^p\ket{\psi_{n,k}} = e^{2\pi i \phi_k / 2^n} \ket{\psi_{n,k}}
\end{equation}

This explains why we don't find even $k$ interesting,
since then we would not get a
cyclic distribution of $2^n$ different phases,
since only odd $k$
are coprime with $2^n$. The exception is $k=0$, since this is the
equal superposition of computational basis states, which we can also
efficiently create. This will be a useful starting point later on to
create addition eigenstates
for odd $k$.

\begin{displaymath}
\ket{\psi_{n,0}} = H^{\otimes n}\ket{0^n}
\end{displaymath}

Suppose we have a certain state $\ket{\psi_{n,k}}$ but we want to get enact
a phase shift $e^{2\pi i l / 2^n}$. We can do this by solving $p=p(s,l)$
in this equation:

\begin{equation}
\label{eqn:psl}
(2s-1)p \equiv l (\bmod 2^n)
\end{equation}

Stipulating $k$ to be odd guarantees that there is a unique solution $p$.

We then apply $A^p$ as follows:

\begin{equation}
\label{eqn:upsilon}
\Upsilon_n(A) \ket{p, \psi_{n,k}} \rightarrow
e^{2\pi i l/2^n} \ket{p, \psi_{n,k}}
\end{equation}

But $p$ is in general hard to solve for, unless $k=1$, in which case
$p = l$. Therefore, we will later see that $\ket{\psi_{n,1}}$ is a desirable
state to have.

Finally, to copy the state $\ket{\psi_{n,k}}$ it suffices to apply the following
operator which only uses subtraction (addition with one addend and the
outcome negated in two's complement representation).

\begin{equation*}
\ket{\psi_{n,k}}^{\otimes m} = W^{-1}\left( \ket{\psi_{n,0}}^\otimes(m-1) \otimes \ket{\psi_{n,k}} \right)
\end{equation*}

where $W$ is the operator on $m$ registers, each of consisting of $n$ qubits,
which adds all the registers into its final register (modulo $2^n$).

\begin{multline}
W : \ket{x_1,\ldots,x_{m-1},x_m} \rightarrow \\
 \ket{x_1,\ldots,x_{m-1},x_1+\ldots+x_m \bmod 2^n}
\end{multline}

Armed with these properties, we're now ready to outline the main steps in KSV.