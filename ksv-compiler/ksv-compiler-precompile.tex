\subsection{Precompiling to the Standard Set}
\label{subsec:precompile}

This is the completely classical preprocessing step for Super-Kitaev which
assumes we can express any $n$-qubit gate as a tensor product of
single- and two-qubit gates.
From there, how do we get down to our universal set?

%\begin{theorem}[Lemma 8.2]
%\label{lemma82}
%Any arbitrary unitary operator $U$ of dimension $d \times d$ can be represented
%as a product of $d(d-2)/2$ matrices of the form:
%
%\begin{equation*}
%\left( \begin{array}{cccccccc}
%     1 & 0      & \cdots & \cdots & \cdots & \cdots & \cdots & \cdots\\
%     0 & 1      & \cdots & \cdots & \cdots & \cdots & \cdots & \cdots\\
%\vdots & \cdots & \ddots & \cdots & \cdots & \cdots & \cdots & \cdots\\
%\vdots & \cdots & \cdots &      a &      b & \cdots & \cdots & \cdots\\
%\vdots & \cdots & \cdots &      c &      d & \cdots & \cdots & \cdots\\
%\vdots & \cdots & \cdots & \cdots & \cdots & \vdots & \cdots & \cdots\\
%\vdots & \cdots & \cdots & \cdots & \cdots & \cdots & 1 & 0\\
%\vdots & \cdots & \cdots & \cdots & \cdots & \cdots & 0 & 1\\
%\end{array}
%\right)
%\end{equation*}
%
%where $\left( \begin{array}{cc}
%a & b\\
%c & d\\
%\end{array} \right) \in SU(2)$
%\end{theorem}
%
%\begin{proof}
%The running time of this algorithm is $O(M^3) \cdot \mathrm{poly}(\log(1/\delta)$.
%\end{proof}

We use the Euler angle decomposition into $X$ and $Z$ rotations,
we can express any
unitary $U$ as follows, with real angles $\phi$, $\gamma$, $\beta$, $\alpha$:

\begin{equation*}
U = e^{i\phi}e^{i(\gamma/2)\sigma^z}e^{i(\beta/2)\sigma^x}e^{i(\alpha/2)\sigma^z}
\end{equation*}
This involves solving four equations in four variables and can be done in
constant time.

Now how do we implement each of the elements of the Euler angle decomposition from
the $\mathcal{Q}$? Using these identities:

\begin{equation*}
e^{i\phi} = \Lambda(e^{i\phi})\sigma^x\Lambda(e^{i\phi})\sigma^x
\end{equation*}
\begin{equation*}
\sigma^x = H\Lambda(e^{i\pi})H
\end{equation*}
\begin{equation*}
e^{i\phi\sigma^z} = \Lambda(e^{-i\phi}\sigma^x\Lambda(e^{i\phi})\sigma^x
\end{equation*}
\begin{equation*}
e^{i\phi\sigma^x} = H e^{i\phi\sigma^z} H
\end{equation*}

In addition,
any controlled operator $\Lambda(U)$ where $U \in SU(2)$ can be simulated
using
$\mathcal{Q} \cup \{ \Lambda(e^{i\phi})$.

We can represent the controlled operator as $\Lambda(U) = \Lambda(e^{i\phi})\Lambda(V)$.
$\Lambda(e^{i\phi})$ is already in our set, so it remains to implement
$\Lambda(V)$, where $V \in SU(2)$. Geometrically, we can decompose any
three-dimensional rotation into two $180^{\circ}$ rotations about appropriate
axes. Using the homomorphism between $SO(3)$ and $SU(2)$ and the fact that
$\sigma^x$ corresponds to a $180^{\circ}$ rotation about the $x$-axis.

\begin{equation*}
V = A\sigma^x A^{-1} B\sigma^x B^{-1}
\end{equation*}
